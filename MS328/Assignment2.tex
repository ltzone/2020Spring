\documentclass[11pt]{article}
\usepackage{fontspec, xunicode, xltxtra}
\setmainfont{Microsoft YaHei}
\usepackage{ctex}
    \usepackage[breakable]{tcolorbox}
    \usepackage{parskip} % Stop auto-indenting (to mimic markdown behaviour)
    
    \usepackage{iftex}
    \ifPDFTeX
    	\usepackage[T1]{fontenc}
    	\usepackage{mathpazo}
    \else
    	\usepackage{fontspec}
    \fi

    % Basic figure setup, for now with no caption control since it's done
    % automatically by Pandoc (which extracts ![](path) syntax from Markdown).
    \usepackage{graphicx}
    % Maintain compatibility with old templates. Remove in nbconvert 6.0
    \let\Oldincludegraphics\includegraphics
    % Ensure that by default, figures have no caption (until we provide a
    % proper Figure object with a Caption API and a way to capture that
    % in the conversion process - todo).
    \usepackage{caption}
    \DeclareCaptionFormat{nocaption}{}
    \captionsetup{format=nocaption,aboveskip=0pt,belowskip=0pt}

    \usepackage[Export]{adjustbox} % Used to constrain images to a maximum size
    \adjustboxset{max size={0.9\linewidth}{0.9\paperheight}}
    \usepackage{float}
    \floatplacement{figure}{H} % forces figures to be placed at the correct location
    \usepackage{xcolor} % Allow colors to be defined
    \usepackage{enumerate} % Needed for markdown enumerations to work
    \usepackage{geometry} % Used to adjust the document margins
    \usepackage{amsmath} % Equations
    \usepackage{amssymb} % Equations
    \usepackage{textcomp} % defines textquotesingle
    % Hack from http://tex.stackexchange.com/a/47451/13684:
    \AtBeginDocument{%
        \def\PYZsq{\textquotesingle}% Upright quotes in Pygmentized code
    }
    \usepackage{upquote} % Upright quotes for verbatim code
    \usepackage{eurosym} % defines \euro
    \usepackage[mathletters]{ucs} % Extended unicode (utf-8) support
    \usepackage{fancyvrb} % verbatim replacement that allows latex
    \usepackage{grffile} % extends the file name processing of package graphics 
                         % to support a larger range
    \makeatletter % fix for grffile with XeLaTeX
    \def\Gread@@xetex#1{%
      \IfFileExists{"\Gin@base".bb}%
      {\Gread@eps{\Gin@base.bb}}%
      {\Gread@@xetex@aux#1}%
    }
    \makeatother

    % The hyperref package gives us a pdf with properly built
    % internal navigation ('pdf bookmarks' for the table of contents,
    % internal cross-reference links, web links for URLs, etc.)
    \usepackage{hyperref}
    % The default LaTeX title has an obnoxious amount of whitespace. By default,
    % titling removes some of it. It also provides customization options.
    \usepackage{titling}
    \usepackage{longtable} % longtable support required by pandoc >1.10
    \usepackage{booktabs}  % table support for pandoc > 1.12.2
    \usepackage[inline]{enumitem} % IRkernel/repr support (it uses the enumerate* environment)
    \usepackage[normalem]{ulem} % ulem is needed to support strikethroughs (\sout)
                                % normalem makes italics be italics, not underlines
    \usepackage{mathrsfs}
    

    
    % Colors for the hyperref package
    \definecolor{urlcolor}{rgb}{0,.145,.698}
    \definecolor{linkcolor}{rgb}{.71,0.21,0.01}
    \definecolor{citecolor}{rgb}{.12,.54,.11}

    % ANSI colors
    \definecolor{ansi-black}{HTML}{3E424D}
    \definecolor{ansi-black-intense}{HTML}{282C36}
    \definecolor{ansi-red}{HTML}{E75C58}
    \definecolor{ansi-red-intense}{HTML}{B22B31}
    \definecolor{ansi-green}{HTML}{00A250}
    \definecolor{ansi-green-intense}{HTML}{007427}
    \definecolor{ansi-yellow}{HTML}{DDB62B}
    \definecolor{ansi-yellow-intense}{HTML}{B27D12}
    \definecolor{ansi-blue}{HTML}{208FFB}
    \definecolor{ansi-blue-intense}{HTML}{0065CA}
    \definecolor{ansi-magenta}{HTML}{D160C4}
    \definecolor{ansi-magenta-intense}{HTML}{A03196}
    \definecolor{ansi-cyan}{HTML}{60C6C8}
    \definecolor{ansi-cyan-intense}{HTML}{258F8F}
    \definecolor{ansi-white}{HTML}{C5C1B4}
    \definecolor{ansi-white-intense}{HTML}{A1A6B2}
    \definecolor{ansi-default-inverse-fg}{HTML}{FFFFFF}
    \definecolor{ansi-default-inverse-bg}{HTML}{000000}

    % commands and environments needed by pandoc snippets
    % extracted from the output of `pandoc -s`
    \providecommand{\tightlist}{%
      \setlength{\itemsep}{0pt}\setlength{\parskip}{0pt}}
    \DefineVerbatimEnvironment{Highlighting}{Verbatim}{commandchars=\\\{\}}
    % Add ',fontsize=\small' for more characters per line
    \newenvironment{Shaded}{}{}
    \newcommand{\KeywordTok}[1]{\textcolor[rgb]{0.00,0.44,0.13}{\textbf{{#1}}}}
    \newcommand{\DataTypeTok}[1]{\textcolor[rgb]{0.56,0.13,0.00}{{#1}}}
    \newcommand{\DecValTok}[1]{\textcolor[rgb]{0.25,0.63,0.44}{{#1}}}
    \newcommand{\BaseNTok}[1]{\textcolor[rgb]{0.25,0.63,0.44}{{#1}}}
    \newcommand{\FloatTok}[1]{\textcolor[rgb]{0.25,0.63,0.44}{{#1}}}
    \newcommand{\CharTok}[1]{\textcolor[rgb]{0.25,0.44,0.63}{{#1}}}
    \newcommand{\StringTok}[1]{\textcolor[rgb]{0.25,0.44,0.63}{{#1}}}
    \newcommand{\CommentTok}[1]{\textcolor[rgb]{0.38,0.63,0.69}{\textit{{#1}}}}
    \newcommand{\OtherTok}[1]{\textcolor[rgb]{0.00,0.44,0.13}{{#1}}}
    \newcommand{\AlertTok}[1]{\textcolor[rgb]{1.00,0.00,0.00}{\textbf{{#1}}}}
    \newcommand{\FunctionTok}[1]{\textcolor[rgb]{0.02,0.16,0.49}{{#1}}}
    \newcommand{\RegionMarkerTok}[1]{{#1}}
    \newcommand{\ErrorTok}[1]{\textcolor[rgb]{1.00,0.00,0.00}{\textbf{{#1}}}}
    \newcommand{\NormalTok}[1]{{#1}}
    
    % Additional commands for more recent versions of Pandoc
    \newcommand{\ConstantTok}[1]{\textcolor[rgb]{0.53,0.00,0.00}{{#1}}}
    \newcommand{\SpecialCharTok}[1]{\textcolor[rgb]{0.25,0.44,0.63}{{#1}}}
    \newcommand{\VerbatimStringTok}[1]{\textcolor[rgb]{0.25,0.44,0.63}{{#1}}}
    \newcommand{\SpecialStringTok}[1]{\textcolor[rgb]{0.73,0.40,0.53}{{#1}}}
    \newcommand{\ImportTok}[1]{{#1}}
    \newcommand{\DocumentationTok}[1]{\textcolor[rgb]{0.73,0.13,0.13}{\textit{{#1}}}}
    \newcommand{\AnnotationTok}[1]{\textcolor[rgb]{0.38,0.63,0.69}{\textbf{\textit{{#1}}}}}
    \newcommand{\CommentVarTok}[1]{\textcolor[rgb]{0.38,0.63,0.69}{\textbf{\textit{{#1}}}}}
    \newcommand{\VariableTok}[1]{\textcolor[rgb]{0.10,0.09,0.49}{{#1}}}
    \newcommand{\ControlFlowTok}[1]{\textcolor[rgb]{0.00,0.44,0.13}{\textbf{{#1}}}}
    \newcommand{\OperatorTok}[1]{\textcolor[rgb]{0.40,0.40,0.40}{{#1}}}
    \newcommand{\BuiltInTok}[1]{{#1}}
    \newcommand{\ExtensionTok}[1]{{#1}}
    \newcommand{\PreprocessorTok}[1]{\textcolor[rgb]{0.74,0.48,0.00}{{#1}}}
    \newcommand{\AttributeTok}[1]{\textcolor[rgb]{0.49,0.56,0.16}{{#1}}}
    \newcommand{\InformationTok}[1]{\textcolor[rgb]{0.38,0.63,0.69}{\textbf{\textit{{#1}}}}}
    \newcommand{\WarningTok}[1]{\textcolor[rgb]{0.38,0.63,0.69}{\textbf{\textit{{#1}}}}}
    
    
    % Define a nice break command that doesn't care if a line doesn't already
    % exist.
    \def\br{\hspace*{\fill} \\* }
    % Math Jax compatibility definitions
    \def\gt{>}
    \def\lt{<}
    \let\Oldtex\TeX
    \let\Oldlatex\LaTeX
    \renewcommand{\TeX}{\textrm{\Oldtex}}
    \renewcommand{\LaTeX}{\textrm{\Oldlatex}}
    % Document parameters
    % Document title
    \title{MS328 Assignment2}
    \author{周李韬 518030910407}
    
    
    
    
    
% Pygments definitions
\makeatletter
\def\PY@reset{\let\PY@it=\relax \let\PY@bf=\relax%
    \let\PY@ul=\relax \let\PY@tc=\relax%
    \let\PY@bc=\relax \let\PY@ff=\relax}
\def\PY@tok#1{\csname PY@tok@#1\endcsname}
\def\PY@toks#1+{\ifx\relax#1\empty\else%
    \PY@tok{#1}\expandafter\PY@toks\fi}
\def\PY@do#1{\PY@bc{\PY@tc{\PY@ul{%
    \PY@it{\PY@bf{\PY@ff{#1}}}}}}}
\def\PY#1#2{\PY@reset\PY@toks#1+\relax+\PY@do{#2}}

\expandafter\def\csname PY@tok@w\endcsname{\def\PY@tc##1{\textcolor[rgb]{0.73,0.73,0.73}{##1}}}
\expandafter\def\csname PY@tok@c\endcsname{\let\PY@it=\textit\def\PY@tc##1{\textcolor[rgb]{0.25,0.50,0.50}{##1}}}
\expandafter\def\csname PY@tok@cp\endcsname{\def\PY@tc##1{\textcolor[rgb]{0.74,0.48,0.00}{##1}}}
\expandafter\def\csname PY@tok@k\endcsname{\let\PY@bf=\textbf\def\PY@tc##1{\textcolor[rgb]{0.00,0.50,0.00}{##1}}}
\expandafter\def\csname PY@tok@kp\endcsname{\def\PY@tc##1{\textcolor[rgb]{0.00,0.50,0.00}{##1}}}
\expandafter\def\csname PY@tok@kt\endcsname{\def\PY@tc##1{\textcolor[rgb]{0.69,0.00,0.25}{##1}}}
\expandafter\def\csname PY@tok@o\endcsname{\def\PY@tc##1{\textcolor[rgb]{0.40,0.40,0.40}{##1}}}
\expandafter\def\csname PY@tok@ow\endcsname{\let\PY@bf=\textbf\def\PY@tc##1{\textcolor[rgb]{0.67,0.13,1.00}{##1}}}
\expandafter\def\csname PY@tok@nb\endcsname{\def\PY@tc##1{\textcolor[rgb]{0.00,0.50,0.00}{##1}}}
\expandafter\def\csname PY@tok@nf\endcsname{\def\PY@tc##1{\textcolor[rgb]{0.00,0.00,1.00}{##1}}}
\expandafter\def\csname PY@tok@nc\endcsname{\let\PY@bf=\textbf\def\PY@tc##1{\textcolor[rgb]{0.00,0.00,1.00}{##1}}}
\expandafter\def\csname PY@tok@nn\endcsname{\let\PY@bf=\textbf\def\PY@tc##1{\textcolor[rgb]{0.00,0.00,1.00}{##1}}}
\expandafter\def\csname PY@tok@ne\endcsname{\let\PY@bf=\textbf\def\PY@tc##1{\textcolor[rgb]{0.82,0.25,0.23}{##1}}}
\expandafter\def\csname PY@tok@nv\endcsname{\def\PY@tc##1{\textcolor[rgb]{0.10,0.09,0.49}{##1}}}
\expandafter\def\csname PY@tok@no\endcsname{\def\PY@tc##1{\textcolor[rgb]{0.53,0.00,0.00}{##1}}}
\expandafter\def\csname PY@tok@nl\endcsname{\def\PY@tc##1{\textcolor[rgb]{0.63,0.63,0.00}{##1}}}
\expandafter\def\csname PY@tok@ni\endcsname{\let\PY@bf=\textbf\def\PY@tc##1{\textcolor[rgb]{0.60,0.60,0.60}{##1}}}
\expandafter\def\csname PY@tok@na\endcsname{\def\PY@tc##1{\textcolor[rgb]{0.49,0.56,0.16}{##1}}}
\expandafter\def\csname PY@tok@nt\endcsname{\let\PY@bf=\textbf\def\PY@tc##1{\textcolor[rgb]{0.00,0.50,0.00}{##1}}}
\expandafter\def\csname PY@tok@nd\endcsname{\def\PY@tc##1{\textcolor[rgb]{0.67,0.13,1.00}{##1}}}
\expandafter\def\csname PY@tok@s\endcsname{\def\PY@tc##1{\textcolor[rgb]{0.73,0.13,0.13}{##1}}}
\expandafter\def\csname PY@tok@sd\endcsname{\let\PY@it=\textit\def\PY@tc##1{\textcolor[rgb]{0.73,0.13,0.13}{##1}}}
\expandafter\def\csname PY@tok@si\endcsname{\let\PY@bf=\textbf\def\PY@tc##1{\textcolor[rgb]{0.73,0.40,0.53}{##1}}}
\expandafter\def\csname PY@tok@se\endcsname{\let\PY@bf=\textbf\def\PY@tc##1{\textcolor[rgb]{0.73,0.40,0.13}{##1}}}
\expandafter\def\csname PY@tok@sr\endcsname{\def\PY@tc##1{\textcolor[rgb]{0.73,0.40,0.53}{##1}}}
\expandafter\def\csname PY@tok@ss\endcsname{\def\PY@tc##1{\textcolor[rgb]{0.10,0.09,0.49}{##1}}}
\expandafter\def\csname PY@tok@sx\endcsname{\def\PY@tc##1{\textcolor[rgb]{0.00,0.50,0.00}{##1}}}
\expandafter\def\csname PY@tok@m\endcsname{\def\PY@tc##1{\textcolor[rgb]{0.40,0.40,0.40}{##1}}}
\expandafter\def\csname PY@tok@gh\endcsname{\let\PY@bf=\textbf\def\PY@tc##1{\textcolor[rgb]{0.00,0.00,0.50}{##1}}}
\expandafter\def\csname PY@tok@gu\endcsname{\let\PY@bf=\textbf\def\PY@tc##1{\textcolor[rgb]{0.50,0.00,0.50}{##1}}}
\expandafter\def\csname PY@tok@gd\endcsname{\def\PY@tc##1{\textcolor[rgb]{0.63,0.00,0.00}{##1}}}
\expandafter\def\csname PY@tok@gi\endcsname{\def\PY@tc##1{\textcolor[rgb]{0.00,0.63,0.00}{##1}}}
\expandafter\def\csname PY@tok@gr\endcsname{\def\PY@tc##1{\textcolor[rgb]{1.00,0.00,0.00}{##1}}}
\expandafter\def\csname PY@tok@ge\endcsname{\let\PY@it=\textit}
\expandafter\def\csname PY@tok@gs\endcsname{\let\PY@bf=\textbf}
\expandafter\def\csname PY@tok@gp\endcsname{\let\PY@bf=\textbf\def\PY@tc##1{\textcolor[rgb]{0.00,0.00,0.50}{##1}}}
\expandafter\def\csname PY@tok@go\endcsname{\def\PY@tc##1{\textcolor[rgb]{0.53,0.53,0.53}{##1}}}
\expandafter\def\csname PY@tok@gt\endcsname{\def\PY@tc##1{\textcolor[rgb]{0.00,0.27,0.87}{##1}}}
\expandafter\def\csname PY@tok@err\endcsname{\def\PY@bc##1{\setlength{\fboxsep}{0pt}\fcolorbox[rgb]{1.00,0.00,0.00}{1,1,1}{\strut ##1}}}
\expandafter\def\csname PY@tok@kc\endcsname{\let\PY@bf=\textbf\def\PY@tc##1{\textcolor[rgb]{0.00,0.50,0.00}{##1}}}
\expandafter\def\csname PY@tok@kd\endcsname{\let\PY@bf=\textbf\def\PY@tc##1{\textcolor[rgb]{0.00,0.50,0.00}{##1}}}
\expandafter\def\csname PY@tok@kn\endcsname{\let\PY@bf=\textbf\def\PY@tc##1{\textcolor[rgb]{0.00,0.50,0.00}{##1}}}
\expandafter\def\csname PY@tok@kr\endcsname{\let\PY@bf=\textbf\def\PY@tc##1{\textcolor[rgb]{0.00,0.50,0.00}{##1}}}
\expandafter\def\csname PY@tok@bp\endcsname{\def\PY@tc##1{\textcolor[rgb]{0.00,0.50,0.00}{##1}}}
\expandafter\def\csname PY@tok@fm\endcsname{\def\PY@tc##1{\textcolor[rgb]{0.00,0.00,1.00}{##1}}}
\expandafter\def\csname PY@tok@vc\endcsname{\def\PY@tc##1{\textcolor[rgb]{0.10,0.09,0.49}{##1}}}
\expandafter\def\csname PY@tok@vg\endcsname{\def\PY@tc##1{\textcolor[rgb]{0.10,0.09,0.49}{##1}}}
\expandafter\def\csname PY@tok@vi\endcsname{\def\PY@tc##1{\textcolor[rgb]{0.10,0.09,0.49}{##1}}}
\expandafter\def\csname PY@tok@vm\endcsname{\def\PY@tc##1{\textcolor[rgb]{0.10,0.09,0.49}{##1}}}
\expandafter\def\csname PY@tok@sa\endcsname{\def\PY@tc##1{\textcolor[rgb]{0.73,0.13,0.13}{##1}}}
\expandafter\def\csname PY@tok@sb\endcsname{\def\PY@tc##1{\textcolor[rgb]{0.73,0.13,0.13}{##1}}}
\expandafter\def\csname PY@tok@sc\endcsname{\def\PY@tc##1{\textcolor[rgb]{0.73,0.13,0.13}{##1}}}
\expandafter\def\csname PY@tok@dl\endcsname{\def\PY@tc##1{\textcolor[rgb]{0.73,0.13,0.13}{##1}}}
\expandafter\def\csname PY@tok@s2\endcsname{\def\PY@tc##1{\textcolor[rgb]{0.73,0.13,0.13}{##1}}}
\expandafter\def\csname PY@tok@sh\endcsname{\def\PY@tc##1{\textcolor[rgb]{0.73,0.13,0.13}{##1}}}
\expandafter\def\csname PY@tok@s1\endcsname{\def\PY@tc##1{\textcolor[rgb]{0.73,0.13,0.13}{##1}}}
\expandafter\def\csname PY@tok@mb\endcsname{\def\PY@tc##1{\textcolor[rgb]{0.40,0.40,0.40}{##1}}}
\expandafter\def\csname PY@tok@mf\endcsname{\def\PY@tc##1{\textcolor[rgb]{0.40,0.40,0.40}{##1}}}
\expandafter\def\csname PY@tok@mh\endcsname{\def\PY@tc##1{\textcolor[rgb]{0.40,0.40,0.40}{##1}}}
\expandafter\def\csname PY@tok@mi\endcsname{\def\PY@tc##1{\textcolor[rgb]{0.40,0.40,0.40}{##1}}}
\expandafter\def\csname PY@tok@il\endcsname{\def\PY@tc##1{\textcolor[rgb]{0.40,0.40,0.40}{##1}}}
\expandafter\def\csname PY@tok@mo\endcsname{\def\PY@tc##1{\textcolor[rgb]{0.40,0.40,0.40}{##1}}}
\expandafter\def\csname PY@tok@ch\endcsname{\let\PY@it=\textit\def\PY@tc##1{\textcolor[rgb]{0.25,0.50,0.50}{##1}}}
\expandafter\def\csname PY@tok@cm\endcsname{\let\PY@it=\textit\def\PY@tc##1{\textcolor[rgb]{0.25,0.50,0.50}{##1}}}
\expandafter\def\csname PY@tok@cpf\endcsname{\let\PY@it=\textit\def\PY@tc##1{\textcolor[rgb]{0.25,0.50,0.50}{##1}}}
\expandafter\def\csname PY@tok@c1\endcsname{\let\PY@it=\textit\def\PY@tc##1{\textcolor[rgb]{0.25,0.50,0.50}{##1}}}
\expandafter\def\csname PY@tok@cs\endcsname{\let\PY@it=\textit\def\PY@tc##1{\textcolor[rgb]{0.25,0.50,0.50}{##1}}}

\def\PYZbs{\char`\\}
\def\PYZus{\char`\_}
\def\PYZob{\char`\{}
\def\PYZcb{\char`\}}
\def\PYZca{\char`\^}
\def\PYZam{\char`\&}
\def\PYZlt{\char`\<}
\def\PYZgt{\char`\>}
\def\PYZsh{\char`\#}
\def\PYZpc{\char`\%}
\def\PYZdl{\char`\$}
\def\PYZhy{\char`\-}
\def\PYZsq{\char`\'}
\def\PYZdq{\char`\"}
\def\PYZti{\char`\~}
% for compatibility with earlier versions
\def\PYZat{@}
\def\PYZlb{[}
\def\PYZrb{]}
\makeatother


    % For linebreaks inside Verbatim environment from package fancyvrb. 
    \makeatletter
        \newbox\Wrappedcontinuationbox 
        \newbox\Wrappedvisiblespacebox 
        \newcommand*\Wrappedvisiblespace {\textcolor{red}{\textvisiblespace}} 
        \newcommand*\Wrappedcontinuationsymbol {\textcolor{red}{\llap{\tiny$\m@th\hookrightarrow$}}} 
        \newcommand*\Wrappedcontinuationindent {3ex } 
        \newcommand*\Wrappedafterbreak {\kern\Wrappedcontinuationindent\copy\Wrappedcontinuationbox} 
        % Take advantage of the already applied Pygments mark-up to insert 
        % potential linebreaks for TeX processing. 
        %        {, <, #, %, $, ' and ": go to next line. 
        %        _, }, ^, &, >, - and ~: stay at end of broken line. 
        % Use of \textquotesingle for straight quote. 
        \newcommand*\Wrappedbreaksatspecials {% 
            \def\PYGZus{\discretionary{\char`\_}{\Wrappedafterbreak}{\char`\_}}% 
            \def\PYGZob{\discretionary{}{\Wrappedafterbreak\char`\{}{\char`\{}}% 
            \def\PYGZcb{\discretionary{\char`\}}{\Wrappedafterbreak}{\char`\}}}% 
            \def\PYGZca{\discretionary{\char`\^}{\Wrappedafterbreak}{\char`\^}}% 
            \def\PYGZam{\discretionary{\char`\&}{\Wrappedafterbreak}{\char`\&}}% 
            \def\PYGZlt{\discretionary{}{\Wrappedafterbreak\char`\<}{\char`\<}}% 
            \def\PYGZgt{\discretionary{\char`\>}{\Wrappedafterbreak}{\char`\>}}% 
            \def\PYGZsh{\discretionary{}{\Wrappedafterbreak\char`\#}{\char`\#}}% 
            \def\PYGZpc{\discretionary{}{\Wrappedafterbreak\char`\%}{\char`\%}}% 
            \def\PYGZdl{\discretionary{}{\Wrappedafterbreak\char`\$}{\char`\$}}% 
            \def\PYGZhy{\discretionary{\char`\-}{\Wrappedafterbreak}{\char`\-}}% 
            \def\PYGZsq{\discretionary{}{\Wrappedafterbreak\textquotesingle}{\textquotesingle}}% 
            \def\PYGZdq{\discretionary{}{\Wrappedafterbreak\char`\"}{\char`\"}}% 
            \def\PYGZti{\discretionary{\char`\~}{\Wrappedafterbreak}{\char`\~}}% 
        } 
        % Some characters . , ; ? ! / are not pygmentized. 
        % This macro makes them "active" and they will insert potential linebreaks 
        \newcommand*\Wrappedbreaksatpunct {% 
            \lccode`\~`\.\lowercase{\def~}{\discretionary{\hbox{\char`\.}}{\Wrappedafterbreak}{\hbox{\char`\.}}}% 
            \lccode`\~`\,\lowercase{\def~}{\discretionary{\hbox{\char`\,}}{\Wrappedafterbreak}{\hbox{\char`\,}}}% 
            \lccode`\~`\;\lowercase{\def~}{\discretionary{\hbox{\char`\;}}{\Wrappedafterbreak}{\hbox{\char`\;}}}% 
            \lccode`\~`\:\lowercase{\def~}{\discretionary{\hbox{\char`\:}}{\Wrappedafterbreak}{\hbox{\char`\:}}}% 
            \lccode`\~`\?\lowercase{\def~}{\discretionary{\hbox{\char`\?}}{\Wrappedafterbreak}{\hbox{\char`\?}}}% 
            \lccode`\~`\!\lowercase{\def~}{\discretionary{\hbox{\char`\!}}{\Wrappedafterbreak}{\hbox{\char`\!}}}% 
            \lccode`\~`\/\lowercase{\def~}{\discretionary{\hbox{\char`\/}}{\Wrappedafterbreak}{\hbox{\char`\/}}}% 
            \catcode`\.\active
            \catcode`\,\active 
            \catcode`\;\active
            \catcode`\:\active
            \catcode`\?\active
            \catcode`\!\active
            \catcode`\/\active 
            \lccode`\~`\~ 	
        }
    \makeatother

    \let\OriginalVerbatim=\Verbatim
    \makeatletter
    \renewcommand{\Verbatim}[1][1]{%
        %\parskip\z@skip
        \sbox\Wrappedcontinuationbox {\Wrappedcontinuationsymbol}%
        \sbox\Wrappedvisiblespacebox {\FV@SetupFont\Wrappedvisiblespace}%
        \def\FancyVerbFormatLine ##1{\hsize\linewidth
            \vtop{\raggedright\hyphenpenalty\z@\exhyphenpenalty\z@
                \doublehyphendemerits\z@\finalhyphendemerits\z@
                \strut ##1\strut}%
        }%
        % If the linebreak is at a space, the latter will be displayed as visible
        % space at end of first line, and a continuation symbol starts next line.
        % Stretch/shrink are however usually zero for typewriter font.
        \def\FV@Space {%
            \nobreak\hskip\z@ plus\fontdimen3\font minus\fontdimen4\font
            \discretionary{\copy\Wrappedvisiblespacebox}{\Wrappedafterbreak}
            {\kern\fontdimen2\font}%
        }%
        
        % Allow breaks at special characters using \PYG... macros.
        \Wrappedbreaksatspecials
        % Breaks at punctuation characters . , ; ? ! and / need catcode=\active 	
        \OriginalVerbatim[#1,codes*=\Wrappedbreaksatpunct]%
    }
    \makeatother

    % Exact colors from NB
    \definecolor{incolor}{HTML}{303F9F}
    \definecolor{outcolor}{HTML}{D84315}
    \definecolor{cellborder}{HTML}{CFCFCF}
    \definecolor{cellbackground}{HTML}{F7F7F7}
    
    % prompt
    \makeatletter
    \newcommand{\boxspacing}{\kern\kvtcb@left@rule\kern\kvtcb@boxsep}
    \makeatother
    \newcommand{\prompt}[4]{
        \ttfamily\llap{{\color{#2}[#3]:\hspace{3pt}#4}}\vspace{-\baselineskip}
    }
    

    
    % Prevent overflowing lines due to hard-to-break entities
    \sloppy 
    % Setup hyperref package
    \hypersetup{
      breaklinks=true,  % so long urls are correctly broken across lines
      colorlinks=true,
      urlcolor=urlcolor,
      linkcolor=linkcolor,
      citecolor=citecolor,
      }
    % Slightly bigger margins than the latex defaults
    
    \geometry{verbose,tmargin=1in,bmargin=1in,lmargin=1in,rmargin=1in}
    
    

\begin{document}
    
    \maketitle
    
    

    
    逻辑回归模型与实现

\begin{enumerate}
\def\labelenumi{\arabic{enumi}.}
\tightlist
\item
  学习一种优化算法,在你熟悉的编程语言中实现逻辑回归的算法;
\item
  基于逻辑回归对某个真实数据进行分析。
\end{enumerate}

\section{数据处理}\label{ux6570ux636eux5904ux7406}

我们采用
\emph{\href{https://archive.ics.uci.edu/ml/datasets/Breast+Cancer+Wisconsin+(Diagnostic)}{Breast
Cancer Wisconsin (Diagnostic) Data Set}} 数据集进行逻辑回归分类.
该数据集中采集了569位病人的胸部细胞特征信息. 胸部细胞共有十类特征,
如细胞半径,灰度,细胞面积等, 均为实数值. 每一类特征具有平均值, 标准差,
极值三个域. 即每位病人有30个数值信息.
每位病人被分类为患癌(M=malignant)和健康(B=benign). 数据集中,
共有357个健康样本, 212个患癌样本. 本实验中,
我们取十种特征的平均值作为样本特征进行回归分类.

下面我们读取对应数据.

    \begin{tcolorbox}[breakable, size=fbox, boxrule=1pt, pad at break*=1mm,colback=cellbackground, colframe=cellborder]
\prompt{In}{incolor}{1}{\boxspacing}
\begin{Verbatim}[commandchars=\\\{\}]
\PY{k+kn}{import} \PY{n+nn}{numpy} \PY{k}{as} \PY{n+nn}{np}
\PY{k+kn}{import} \PY{n+nn}{pandas} \PY{k}{as} \PY{n+nn}{pd}
\PY{k+kn}{import} \PY{n+nn}{math}
\PY{n}{data} \PY{o}{=} \PY{n}{pd}\PY{o}{.}\PY{n}{read\PYZus{}csv}\PY{p}{(}\PY{l+s+s1}{\PYZsq{}}\PY{l+s+s1}{wdbc.data}\PY{l+s+s1}{\PYZsq{}}\PY{p}{,}\PY{n}{header}\PY{o}{=}\PY{k+kc}{None}\PY{p}{,}\PY{n}{index\PYZus{}col}\PY{o}{=}\PY{l+m+mi}{0}\PY{p}{,}\PY{n}{usecols}\PY{o}{=}\PY{p}{[}\PY{l+m+mi}{0}\PY{p}{,}\PY{l+m+mi}{1}\PY{p}{]}\PY{o}{+}\PY{p}{[}\PY{l+m+mi}{2}\PY{o}{+}\PY{l+m+mi}{3}\PY{o}{*}\PY{n}{i} \PY{k}{for} \PY{n}{i} \PY{o+ow}{in} \PY{n+nb}{range}\PY{p}{(}\PY{l+m+mi}{10}\PY{p}{)}\PY{p}{]}\PY{p}{)}
\PY{n}{data}
\end{Verbatim}
\end{tcolorbox}

            \begin{tcolorbox}[breakable, size=fbox, boxrule=.5pt, pad at break*=1mm, opacityfill=0]
\prompt{Out}{outcolor}{1}{\boxspacing}
\begin{Verbatim}[commandchars=\\\{\}]
         1      2       5        8        11     14       17       20     23  \textbackslash{}
0
842302    M  17.99  1001.0  0.30010  0.07871  8.589  0.04904  0.03003  17.33
842517    M  20.57  1326.0  0.08690  0.05667  3.398  0.01308  0.01389  23.41
84300903  M  19.69  1203.0  0.19740  0.05999  4.585  0.04006  0.02250  25.53
84348301  M  11.42   386.1  0.24140  0.09744  3.445  0.07458  0.05963  26.50
84358402  M  20.29  1297.0  0.19800  0.05883  5.438  0.02461  0.01756  16.67
{\ldots}      ..    {\ldots}     {\ldots}      {\ldots}      {\ldots}    {\ldots}      {\ldots}      {\ldots}    {\ldots}
926424    M  21.56  1479.0  0.24390  0.05623  7.673  0.02891  0.01114  26.40
926682    M  20.13  1261.0  0.14400  0.05533  5.203  0.02423  0.01898  38.25
926954    M  16.60   858.1  0.09251  0.05648  3.425  0.03731  0.01318  34.12
927241    M  20.60  1265.0  0.35140  0.07016  5.772  0.06158  0.02324  39.42
92751     B   7.76   181.0  0.00000  0.05884  2.548  0.00466  0.02676  30.37

               26      29
0
842302    0.16220  0.2654
842517    0.12380  0.1860
84300903  0.14440  0.2430
84348301  0.20980  0.2575
84358402  0.13740  0.1625
{\ldots}           {\ldots}     {\ldots}
926424    0.14100  0.2216
926682    0.11660  0.1628
926954    0.11390  0.1418
927241    0.16500  0.2650
92751     0.08996  0.0000

[569 rows x 11 columns]
\end{Verbatim}
\end{tcolorbox}
        
    我们从数据集中提取自变量与因变量.

    \begin{tcolorbox}[breakable, size=fbox, boxrule=1pt, pad at break*=1mm,colback=cellbackground, colframe=cellborder]
\prompt{In}{incolor}{2}{\boxspacing}
\begin{Verbatim}[commandchars=\\\{\}]
\PY{k}{def} \PY{n+nf}{read\PYZus{}sig}\PY{p}{(}\PY{n}{data}\PY{p}{)}\PY{p}{:}
    \PY{n}{x} \PY{o}{=} \PY{p}{[}\PY{p}{]}
    \PY{n}{y} \PY{o}{=} \PY{p}{[}\PY{p}{]}
    \PY{k}{for} \PY{n}{line} \PY{o+ow}{in} \PY{n}{data}\PY{o}{.}\PY{n}{values}\PY{p}{:}
        \PY{n}{x}\PY{o}{.}\PY{n}{append}\PY{p}{(}\PY{n}{line}\PY{p}{[}\PY{l+m+mi}{1}\PY{p}{:}\PY{p}{]}\PY{p}{)}
        \PY{k}{if} \PY{n}{line}\PY{p}{[}\PY{l+m+mi}{0}\PY{p}{]} \PY{o}{==} \PY{l+s+s1}{\PYZsq{}}\PY{l+s+s1}{M}\PY{l+s+s1}{\PYZsq{}}\PY{p}{:}
            \PY{n}{y}\PY{o}{.}\PY{n}{append}\PY{p}{(}\PY{p}{[}\PY{l+m+mi}{1}\PY{p}{]}\PY{p}{)}
        \PY{k}{else}\PY{p}{:}
            \PY{n}{y}\PY{o}{.}\PY{n}{append}\PY{p}{(}\PY{p}{[}\PY{l+m+mi}{0}\PY{p}{]}\PY{p}{)}
    \PY{n}{x} \PY{o}{=} \PY{n}{np}\PY{o}{.}\PY{n}{array}\PY{p}{(}\PY{n}{x}\PY{p}{)}
    \PY{n}{y} \PY{o}{=} \PY{n}{np}\PY{o}{.}\PY{n}{array}\PY{p}{(}\PY{n}{y}\PY{p}{)}
    \PY{k}{return} \PY{n}{x}\PY{p}{,}\PY{n}{y}
\end{Verbatim}
\end{tcolorbox}

    由于十类特征各不相同, 我们需要对数据做一些处理.
经过尝试发现\[x = \frac{x - \bar{x}}{x_{max} - x_{min}}\]
最适合逻辑回归模型的处理方式。首先,样本特征与样本均值的差可以保证数据可以均匀分布在0点两侧,便于计算。进一步,将差值与特征值域作商可以帮助我们在结论中得到各指标之间的相对关系。

    \begin{tcolorbox}[breakable, size=fbox, boxrule=1pt, pad at break*=1mm,colback=cellbackground, colframe=cellborder]
\prompt{In}{incolor}{37}{\boxspacing}
\begin{Verbatim}[commandchars=\\\{\}]
\PY{c+c1}{\PYZsh{} def normalize2(x):}
\PY{c+c1}{\PYZsh{}     return ((x \PYZhy{} x.min(axis=0))  / (x.max(axis=0) \PYZhy{} x.min(axis=0)))}
\PY{c+c1}{\PYZsh{} def normalize3(x):}
\PY{c+c1}{\PYZsh{}     return (x  / (x.max(axis=0) ))}
\PY{k}{def} \PY{n+nf}{normalize}\PY{p}{(}\PY{n}{x}\PY{p}{)}\PY{p}{:}
    \PY{k}{return} \PY{p}{(}\PY{p}{(}\PY{n}{x}\PY{o}{\PYZhy{}}\PY{n}{x}\PY{o}{.}\PY{n}{mean}\PY{p}{(}\PY{n}{axis}\PY{o}{=}\PY{l+m+mi}{0}\PY{p}{)}\PY{p}{)} \PY{o}{/}\PY{p}{(}\PY{n}{x}\PY{o}{.}\PY{n}{max}\PY{p}{(}\PY{n}{axis}\PY{o}{=}\PY{l+m+mi}{0}\PY{p}{)}\PY{o}{\PYZhy{}}\PY{n}{x}\PY{o}{.}\PY{n}{min}\PY{p}{(}\PY{n}{axis}\PY{o}{=}\PY{l+m+mi}{0}\PY{p}{)}\PY{p}{)}\PY{p}{)}
\end{Verbatim}
\end{tcolorbox}

    首先,我们用python
sklearn库中的逻辑回归模型对数据进行测试。注意到sklearn中的求解算法是"lbfgs"的近似算法,因此可能会和我们梯度下降、牛顿法的结果略有不同。sklearn模块的逻辑回归系数结果如下所示。

    \begin{tcolorbox}[breakable, size=fbox, boxrule=1pt, pad at break*=1mm,colback=cellbackground, colframe=cellborder]
\prompt{In}{incolor}{38}{\boxspacing}
\begin{Verbatim}[commandchars=\\\{\}]
\PY{n}{x1}\PY{p}{,}\PY{n}{y} \PY{o}{=} \PY{n}{read\PYZus{}sig}\PY{p}{(}\PY{n}{data}\PY{p}{)}
\PY{n}{x} \PY{o}{=} \PY{n}{normalize}\PY{p}{(}\PY{n}{x1}\PY{p}{)}
\PY{k+kn}{from} \PY{n+nn}{sklearn}\PY{n+nn}{.}\PY{n+nn}{linear\PYZus{}model} \PY{k+kn}{import} \PY{n}{LogisticRegression}
\PY{n}{log\PYZus{}reg} \PY{o}{=} \PY{n}{LogisticRegression}\PY{p}{(}\PY{p}{)}
\PY{n}{log\PYZus{}reg}\PY{o}{.}\PY{n}{fit}\PY{p}{(}\PY{n}{x}\PY{p}{,}\PY{n}{y}\PY{p}{[}\PY{p}{:}\PY{p}{,}\PY{l+m+mi}{0}\PY{p}{]}\PY{p}{)}
\PY{n+nb}{print}\PY{p}{(}\PY{n}{log\PYZus{}reg}\PY{o}{.}\PY{n}{coef\PYZus{}}\PY{p}{)}
\PY{n+nb}{print}\PY{p}{(}\PY{n}{log\PYZus{}reg}\PY{o}{.}\PY{n}{intercept\PYZus{}}\PY{p}{)}
\end{Verbatim}
\end{tcolorbox}

    \begin{Verbatim}[commandchars=\\\{\}]
[[ 3.84162736  3.21032364  2.83241674 -1.17741758  1.81722502 -0.33158592
   0.10224375  3.14757253  2.2137123   4.90276183]]
[-0.74067841]
    \end{Verbatim}

    \section{逻辑回归模型实现}\label{ux903bux8f91ux56deux5f52ux6a21ux578bux5b9eux73b0}

\subsection{梯度下降法}\label{ux68afux5ea6ux4e0bux964dux6cd5}

逻辑回归损失函数的梯度函数如下所示。 \[
\nabla L(\beta) = \frac{1}{m} \sum_{i=1}^m\left( \frac{e^{X_i^{T}\beta}}{1 + e^{X_i^{T}\beta}} X_i - Y_i X_i \right)= \frac{1}{m} \sum_{i=1}^m \left( \frac{1}{1 + e^{ - X_i^{T}\beta}} X_i - Y_i X_i \right)
\]

为减小运算量,避免溢出,我们在梯度函数的实现上选择后一种形式。

    \begin{tcolorbox}[breakable, size=fbox, boxrule=1pt, pad at break*=1mm,colback=cellbackground, colframe=cellborder]
\prompt{In}{incolor}{39}{\boxspacing}
\begin{Verbatim}[commandchars=\\\{\}]
\PY{c+c1}{\PYZsh{} 损失梯度函数}
\PY{k}{def} \PY{n+nf}{logit\PYZus{}der1}\PY{p}{(}\PY{n}{x}\PY{p}{,}\PY{n}{y}\PY{p}{,}\PY{n}{b}\PY{p}{)}\PY{p}{:}
    \PY{n}{x}\PY{o}{=}\PY{n}{np}\PY{o}{.}\PY{n}{array}\PY{p}{(}\PY{n}{x}\PY{p}{,}\PY{n}{dtype}\PY{o}{=}\PY{n}{np}\PY{o}{.}\PY{n}{float64}\PY{p}{)}
    \PY{n}{b}\PY{o}{=}\PY{n}{np}\PY{o}{.}\PY{n}{array}\PY{p}{(}\PY{n}{b}\PY{p}{,}\PY{n}{dtype}\PY{o}{=}\PY{n}{np}\PY{o}{.}\PY{n}{float64}\PY{p}{)}
    \PY{n}{dim} \PY{o}{=} \PY{n}{x}\PY{o}{.}\PY{n}{shape}\PY{p}{[}\PY{l+m+mi}{0}\PY{p}{]}
    \PY{n}{sigmoid} \PY{o}{=} \PY{l+m+mi}{1}\PY{o}{/}\PY{p}{(}\PY{l+m+mi}{1}\PY{o}{+}\PY{n}{np}\PY{o}{.}\PY{n}{exp}\PY{p}{(}\PY{o}{\PYZhy{}} \PY{n}{np}\PY{o}{.}\PY{n}{dot}\PY{p}{(}\PY{n}{x}\PY{p}{,}\PY{n}{b}\PY{p}{)}\PY{p}{)}\PY{p}{)}
    \PY{k}{return} \PY{n+nb}{sum}\PY{p}{(}\PY{p}{[}\PY{p}{(}\PY{n}{sigmoid}\PY{p}{[}\PY{n}{i}\PY{p}{]} \PY{o}{\PYZhy{}} \PY{n}{y}\PY{p}{[}\PY{n}{i}\PY{p}{]}\PY{p}{)}\PY{o}{*}\PY{n}{x}\PY{p}{[}\PY{n}{i}\PY{p}{]} \PY{k}{for} \PY{n}{i} \PY{o+ow}{in} \PY{n+nb}{range}\PY{p}{(}\PY{n}{dim}\PY{p}{)}\PY{p}{]}\PY{p}{)}\PY{o}{/}\PY{n}{dim}

\PY{c+c1}{\PYZsh{} def logit\PYZus{}der1\PYZus{}fail(x,y,b):}
\PY{c+c1}{\PYZsh{}     dim = x.shape[0]}
\PY{c+c1}{\PYZsh{}     return sum([((np.exp(np.vdot(x[i],b))/(np.exp(np.vdot(x[i],b))+1))*x[i] + y[i]*x[i]) for i in range(dim)])/dim}
\end{Verbatim}
\end{tcolorbox}

    为便于观察,我们希望在迭代过程中记录损失函数值。同样,为简化运算,避免正数指数函数运算时的溢出问题,我们也对损失函数做一定变形,通过第二行的式子实现。
\[
\begin{aligned}
L(\beta)&= \frac{1}{m} \sum_{i=1}^m \left( \left(\ln (1+e^{X_i^{T}\beta}\right) - Y_i X_i^{T}\beta \right) \\
&= - \frac{1}{m} \sum_{i=1}^m \left( Y_i \ln \frac{1}{1+e^{-X_i^{T}\beta}} + (1 - Y_i) \ln \left(1 - \frac{1}{1+e^{-X_i^{T}\beta}} \right) \right) \\
\end{aligned}
\]

    \begin{tcolorbox}[breakable, size=fbox, boxrule=1pt, pad at break*=1mm,colback=cellbackground, colframe=cellborder]
\prompt{In}{incolor}{40}{\boxspacing}
\begin{Verbatim}[commandchars=\\\{\}]
\PY{c+c1}{\PYZsh{} 损失函数}
\PY{k}{def} \PY{n+nf}{logit\PYZus{}loss}\PY{p}{(}\PY{n}{x}\PY{p}{,}\PY{n}{y}\PY{p}{,}\PY{n}{b}\PY{p}{)}\PY{p}{:}
    \PY{n}{x}\PY{o}{=}\PY{n}{np}\PY{o}{.}\PY{n}{array}\PY{p}{(}\PY{n}{x}\PY{p}{,}\PY{n}{dtype}\PY{o}{=}\PY{n}{np}\PY{o}{.}\PY{n}{float64}\PY{p}{)}
    \PY{n}{b}\PY{o}{=}\PY{n}{np}\PY{o}{.}\PY{n}{array}\PY{p}{(}\PY{n}{b}\PY{p}{,}\PY{n}{dtype}\PY{o}{=}\PY{n}{np}\PY{o}{.}\PY{n}{float64}\PY{p}{)}
    \PY{n}{sigmoid} \PY{o}{=} \PY{l+m+mi}{1}\PY{o}{/}\PY{p}{(}\PY{l+m+mi}{1}\PY{o}{+}\PY{n}{np}\PY{o}{.}\PY{n}{exp}\PY{p}{(}\PY{o}{\PYZhy{}} \PY{n}{np}\PY{o}{.}\PY{n}{dot}\PY{p}{(}\PY{n}{x}\PY{p}{,}\PY{n}{b}\PY{p}{)}\PY{p}{)}\PY{p}{)}
    \PY{n}{dim} \PY{o}{=} \PY{n}{x}\PY{o}{.}\PY{n}{shape}\PY{p}{[}\PY{l+m+mi}{0}\PY{p}{]}
    \PY{k}{return} \PY{n}{np}\PY{o}{.}\PY{n}{sum}\PY{p}{(}\PY{p}{(}\PY{o}{\PYZhy{}} \PY{n}{y}\PY{o}{*}\PY{n}{np}\PY{o}{.}\PY{n}{log}\PY{p}{(}\PY{n}{sigmoid}\PY{p}{)}\PY{o}{\PYZhy{}} \PY{p}{(}\PY{l+m+mi}{1}\PY{o}{\PYZhy{}}\PY{n}{y}\PY{p}{)}\PY{o}{*}\PY{n}{np}\PY{o}{.}\PY{n}{log}\PY{p}{(}\PY{l+m+mi}{1}\PY{o}{\PYZhy{}}\PY{n}{sigmoid}\PY{p}{)}\PY{p}{)}\PY{p}{)}\PY{o}{/}\PY{n}{dim}

\PY{c+c1}{\PYZsh{} def logit\PYZus{}loss\PYZus{}fail(x,y,b):}
\PY{c+c1}{\PYZsh{}     dim = x.shape[0]}
\PY{c+c1}{\PYZsh{}     return sum([math.log(math.exp(np.vdot(x[i],b))+1) \PYZhy{} y[i] * np.vdot(x[i],b) for i in range(dim)])/dim}
\end{Verbatim}
\end{tcolorbox}

    我们首先尝试选择固定步长进行迭代运算。

    \begin{tcolorbox}[breakable, size=fbox, boxrule=1pt, pad at break*=1mm,colback=cellbackground, colframe=cellborder]
\prompt{In}{incolor}{41}{\boxspacing}
\begin{Verbatim}[commandchars=\\\{\}]
\PY{k}{def} \PY{n+nf}{logit\PYZus{}gradient\PYZus{}descent}\PY{p}{(}\PY{n}{x}\PY{p}{,}\PY{n}{y}\PY{p}{,}\PY{n}{init}\PY{p}{,}\PY{n}{epoch}\PY{p}{)}\PY{p}{:}
    \PY{n}{b} \PY{o}{=} \PY{n}{np}\PY{o}{.}\PY{n}{ones}\PY{p}{(}\PY{n}{x}\PY{o}{.}\PY{n}{shape}\PY{p}{[}\PY{l+m+mi}{0}\PY{p}{]}\PY{p}{)}
    \PY{n}{x1} \PY{o}{=} \PY{n}{np}\PY{o}{.}\PY{n}{c\PYZus{}}\PY{p}{[}\PY{n}{b}\PY{p}{,}\PY{n}{x}\PY{p}{]}              \PY{c+c1}{\PYZsh{} 为自变量加上系数列}
    \PY{n}{beta} \PY{o}{=} \PY{n}{np}\PY{o}{.}\PY{n}{concatenate}\PY{p}{(}\PY{p}{(}\PY{n}{init}\PY{o}{.}\PY{n}{reshape}\PY{p}{(}\PY{l+m+mi}{1}\PY{p}{,}\PY{n}{init}\PY{o}{.}\PY{n}{shape}\PY{p}{[}\PY{l+m+mi}{0}\PY{p}{]}\PY{p}{)}\PY{p}{,}\PY{n}{np}\PY{o}{.}\PY{n}{zeros}\PY{p}{(}\PY{p}{(}\PY{n}{epoch}\PY{p}{,}\PY{n}{x1}\PY{o}{.}\PY{n}{shape}\PY{p}{[}\PY{l+m+mi}{1}\PY{p}{]}\PY{p}{)}\PY{p}{)}\PY{p}{)}\PY{p}{)} \PY{c+c1}{\PYZsh{} 创建迭代记录数组}
    \PY{n}{loss} \PY{o}{=} \PY{n}{np}\PY{o}{.}\PY{n}{zeros}\PY{p}{(}\PY{n}{epoch}\PY{o}{+}\PY{l+m+mi}{1}\PY{p}{)}
    \PY{n}{loss}\PY{p}{[}\PY{l+m+mi}{0}\PY{p}{]} \PY{o}{=} \PY{n}{logit\PYZus{}loss}\PY{p}{(}\PY{n}{x1}\PY{p}{,}\PY{n}{y}\PY{p}{,}\PY{n}{beta}\PY{p}{[}\PY{l+m+mi}{0}\PY{p}{]}\PY{p}{)}
    \PY{k}{for} \PY{n+nb}{iter} \PY{o+ow}{in} \PY{n+nb}{range}\PY{p}{(}\PY{n}{epoch}\PY{p}{)}\PY{p}{:}
        \PY{n}{beta}\PY{p}{[}\PY{n+nb}{iter}\PY{o}{+}\PY{l+m+mi}{1}\PY{p}{]} \PY{o}{=} \PY{n}{beta}\PY{p}{[}\PY{n+nb}{iter}\PY{p}{]} \PY{o}{\PYZhy{}} \PY{l+m+mf}{0.01} \PY{o}{*} \PY{n}{logit\PYZus{}der1}\PY{p}{(}\PY{n}{x1}\PY{p}{,}\PY{n}{y}\PY{p}{,}\PY{n}{beta}\PY{p}{[}\PY{n+nb}{iter}\PY{p}{]}\PY{p}{)}
        \PY{n}{loss}\PY{p}{[}\PY{n+nb}{iter}\PY{o}{+}\PY{l+m+mi}{1}\PY{p}{]} \PY{o}{=} \PY{n}{logit\PYZus{}loss}\PY{p}{(}\PY{n}{x1}\PY{p}{,}\PY{n}{y}\PY{p}{,}\PY{n}{beta}\PY{p}{[}\PY{n+nb}{iter}\PY{o}{+}\PY{l+m+mi}{1}\PY{p}{]}\PY{p}{)}
    \PY{k}{return} \PY{n}{beta}\PY{p}{,}\PY{n}{loss}
\end{Verbatim}
\end{tcolorbox}

    取步长为0.01,初始值为0,迭代1000次,我们发现由于变量维数过多,固定步长下,迭代后期步长过大,存在无法收敛的问题。

    \begin{tcolorbox}[breakable, size=fbox, boxrule=1pt, pad at break*=1mm,colback=cellbackground, colframe=cellborder]
\prompt{In}{incolor}{42}{\boxspacing}
\begin{Verbatim}[commandchars=\\\{\}]
\PY{n}{x1}\PY{p}{,}\PY{n}{y} \PY{o}{=} \PY{n}{read\PYZus{}sig}\PY{p}{(}\PY{n}{data}\PY{p}{)}  \PY{c+c1}{\PYZsh{} 读取数据}
\PY{n}{x} \PY{o}{=} \PY{n}{normalize}\PY{p}{(}\PY{n}{x1}\PY{p}{)}      \PY{c+c1}{\PYZsh{} 数据归一化}
\PY{n}{result}\PY{p}{,}\PY{n}{loss} \PY{o}{=} \PY{n}{logit\PYZus{}gradient\PYZus{}descent}\PY{p}{(}\PY{n}{x}\PY{p}{,}\PY{n}{y}\PY{p}{,}\PY{n}{np}\PY{o}{.}\PY{n}{zeros}\PY{p}{(}\PY{n}{x}\PY{o}{.}\PY{n}{shape}\PY{p}{[}\PY{l+m+mi}{1}\PY{p}{]}\PY{o}{+}\PY{l+m+mi}{1}\PY{p}{)}\PY{p}{,}\PY{l+m+mi}{1000}\PY{p}{)}
\PY{n+nb}{print}\PY{p}{(}\PY{n}{result}\PY{p}{[}\PY{l+m+mi}{1000}\PY{p}{]}\PY{p}{)}    \PY{c+c1}{\PYZsh{} 1000次迭代后的结果}
\end{Verbatim}
\end{tcolorbox}

    \begin{Verbatim}[commandchars=\\\{\}]
[-0.48168394  0.50799971  0.43976882  0.53228799 -0.0179412   0.21863828
  0.15249567 -0.00506235  0.31989157  0.26665304  0.74487394]
    \end{Verbatim}

    \begin{tcolorbox}[breakable, size=fbox, boxrule=1pt, pad at break*=1mm,colback=cellbackground, colframe=cellborder]
\prompt{In}{incolor}{43}{\boxspacing}
\begin{Verbatim}[commandchars=\\\{\}]
\PY{k+kn}{import} \PY{n+nn}{matplotlib}\PY{n+nn}{.}\PY{n+nn}{pyplot} \PY{k}{as} \PY{n+nn}{plt}
\PY{n}{x\PYZus{}ray}\PY{o}{=}\PY{n+nb}{range}\PY{p}{(}\PY{l+m+mi}{0}\PY{p}{,}\PY{l+m+mi}{1001}\PY{p}{)}
\PY{n}{plt}\PY{o}{.}\PY{n}{plot}\PY{p}{(}\PY{n}{x\PYZus{}ray}\PY{p}{,}\PY{n}{result}\PY{p}{)}
\PY{n}{plt}\PY{o}{.}\PY{n}{show}\PY{p}{(}\PY{p}{)}
\PY{c+c1}{\PYZsh{} 迭代过程中beta值的变化,横轴为迭代次数,纵轴为beta[i]的值}
\end{Verbatim}
\end{tcolorbox}

    \begin{center}
    \adjustimage{max size={0.9\linewidth}{0.9\paperheight}}{output_16_0.png}
    \end{center}
    { \hspace*{\fill} \\}
    
    \begin{tcolorbox}[breakable, size=fbox, boxrule=1pt, pad at break*=1mm,colback=cellbackground, colframe=cellborder]
\prompt{In}{incolor}{44}{\boxspacing}
\begin{Verbatim}[commandchars=\\\{\}]
\PY{n}{plt}\PY{o}{.}\PY{n}{plot}\PY{p}{(}\PY{n}{x\PYZus{}ray}\PY{p}{,}\PY{n}{loss}\PY{p}{)}
\PY{n}{plt}\PY{o}{.}\PY{n}{show}\PY{p}{(}\PY{p}{)}
\PY{c+c1}{\PYZsh{} 迭代过程中损失函数值的变化,横轴为迭代次数,纵轴为损失函数值}
\end{Verbatim}
\end{tcolorbox}

    \begin{center}
    \adjustimage{max size={0.9\linewidth}{0.9\paperheight}}{output_17_0.png}
    \end{center}
    { \hspace*{\fill} \\}
    
    \subsection{改进的梯度下降法:Backtracking Line
Search}\label{ux6539ux8fdbux7684ux68afux5ea6ux4e0bux964dux6cd5backtracking-line-search}

我们试图从改进步长选取的角度解决这一问题。注意到有效步长会随着梯度迭代而减小,因此我们设置一个学习率lr,在选定梯度下降方向的情况下,如果在该方向上前进原有的步长使损失函数值不减反增,说明我们的步长需要进一步减小,我们规定每次减小的比例为lr。因此我们在迭代过程中增设一个循环,用于调整步长。

    \begin{tcolorbox}[breakable, size=fbox, boxrule=1pt, pad at break*=1mm,colback=cellbackground, colframe=cellborder]
\prompt{In}{incolor}{12}{\boxspacing}
\begin{Verbatim}[commandchars=\\\{\}]
\PY{k}{def} \PY{n+nf}{logit\PYZus{}line\PYZus{}search}\PY{p}{(}\PY{n}{x}\PY{p}{,}\PY{n}{y}\PY{p}{,}\PY{n}{init}\PY{p}{,}\PY{n}{epoch}\PY{p}{,}\PY{n}{lr}\PY{p}{)}\PY{p}{:}
    \PY{n}{b} \PY{o}{=} \PY{n}{np}\PY{o}{.}\PY{n}{ones}\PY{p}{(}\PY{n}{x}\PY{o}{.}\PY{n}{shape}\PY{p}{[}\PY{l+m+mi}{0}\PY{p}{]}\PY{p}{)}
    \PY{n}{x1} \PY{o}{=} \PY{n}{np}\PY{o}{.}\PY{n}{c\PYZus{}}\PY{p}{[}\PY{n}{b}\PY{p}{,}\PY{n}{x}\PY{p}{]}              \PY{c+c1}{\PYZsh{} 为自变量加上系数列}
    \PY{n}{beta} \PY{o}{=} \PY{n}{np}\PY{o}{.}\PY{n}{concatenate}\PY{p}{(}\PY{p}{(}\PY{n}{init}\PY{o}{.}\PY{n}{reshape}\PY{p}{(}\PY{l+m+mi}{1}\PY{p}{,}\PY{n}{init}\PY{o}{.}\PY{n}{shape}\PY{p}{[}\PY{l+m+mi}{0}\PY{p}{]}\PY{p}{)}\PY{p}{,}\PY{n}{np}\PY{o}{.}\PY{n}{zeros}\PY{p}{(}\PY{p}{(}\PY{n}{epoch}\PY{p}{,}\PY{n}{x1}\PY{o}{.}\PY{n}{shape}\PY{p}{[}\PY{l+m+mi}{1}\PY{p}{]}\PY{p}{)}\PY{p}{)}\PY{p}{)}\PY{p}{)} \PY{c+c1}{\PYZsh{} 准备用于记录迭代结果的数组}
    \PY{n}{loss} \PY{o}{=} \PY{n}{np}\PY{o}{.}\PY{n}{zeros}\PY{p}{(}\PY{n}{epoch}\PY{p}{)}
    \PY{n}{step} \PY{o}{=} \PY{l+m+mf}{0.01}                     \PY{c+c1}{\PYZsh{} 起始步长}
    \PY{k}{for} \PY{n+nb}{iter} \PY{o+ow}{in} \PY{n+nb}{range}\PY{p}{(}\PY{n}{epoch}\PY{p}{)}\PY{p}{:}
        \PY{n}{dx} \PY{o}{=} \PY{n}{logit\PYZus{}der1}\PY{p}{(}\PY{n}{x1}\PY{p}{,}\PY{n}{y}\PY{p}{,}\PY{n}{beta}\PY{p}{[}\PY{n+nb}{iter}\PY{p}{]}\PY{p}{)}
        \PY{n}{loss}\PY{p}{[}\PY{n+nb}{iter}\PY{p}{]} \PY{o}{=} \PY{n}{logit\PYZus{}loss}\PY{p}{(}\PY{n}{x1}\PY{p}{,}\PY{n}{y}\PY{p}{,}\PY{n}{beta}\PY{p}{[}\PY{n+nb}{iter}\PY{p}{]}\PY{p}{)}
        \PY{k}{while} \PY{p}{(}\PY{n}{logit\PYZus{}loss}\PY{p}{(}\PY{n}{x1}\PY{p}{,}\PY{n}{y}\PY{p}{,}\PY{n}{beta}\PY{p}{[}\PY{n+nb}{iter}\PY{p}{]} \PY{o}{\PYZhy{}} \PY{n}{step}\PY{o}{*}\PY{n}{dx}\PY{p}{)} \PY{o}{\PYZgt{}} \PY{n}{loss}\PY{p}{[}\PY{n+nb}{iter}\PY{p}{]} \PY{o}{\PYZhy{}} \PY{l+m+mf}{0.25}\PY{o}{*}\PY{n}{step} \PY{o}{*}\PY{n}{np}\PY{o}{.}\PY{n}{vdot}\PY{p}{(}\PY{n}{dx}\PY{p}{,}\PY{n}{dx}\PY{p}{)}\PY{p}{)}\PY{p}{:}
            \PY{n}{step} \PY{o}{=} \PY{n}{lr} \PY{o}{*} \PY{n}{step}           \PY{c+c1}{\PYZsh{} 在一定条件下调整步长}
        \PY{n}{beta}\PY{p}{[}\PY{n+nb}{iter}\PY{o}{+}\PY{l+m+mi}{1}\PY{p}{]} \PY{o}{=} \PY{n}{beta}\PY{p}{[}\PY{n+nb}{iter}\PY{p}{]} \PY{o}{\PYZhy{}} \PY{n}{step} \PY{o}{*} \PY{n}{logit\PYZus{}der1}\PY{p}{(}\PY{n}{x1}\PY{p}{,}\PY{n}{y}\PY{p}{,}\PY{n}{beta}\PY{p}{[}\PY{n+nb}{iter}\PY{p}{]}\PY{p}{)}
    \PY{k}{return} \PY{n}{beta}\PY{p}{,}\PY{n}{loss}
\end{Verbatim}
\end{tcolorbox}

    我们仍设置初始步长为0.01,迭代过程中步长以0.75的比例减小。同样迭代1000次。

    \begin{tcolorbox}[breakable, size=fbox, boxrule=1pt, pad at break*=1mm,colback=cellbackground, colframe=cellborder]
\prompt{In}{incolor}{13}{\boxspacing}
\begin{Verbatim}[commandchars=\\\{\}]
\PY{n}{x1}\PY{p}{,}\PY{n}{y} \PY{o}{=} \PY{n}{read\PYZus{}sig}\PY{p}{(}\PY{n}{data}\PY{p}{)}
\PY{n}{x} \PY{o}{=} \PY{n}{normalize}\PY{p}{(}\PY{n}{x1}\PY{p}{)}
\PY{n}{ls\PYZus{}result}\PY{p}{,}\PY{n}{ls\PYZus{}loss} \PY{o}{=} \PY{n}{logit\PYZus{}line\PYZus{}search}\PY{p}{(}\PY{n}{x}\PY{p}{,}\PY{n}{y}\PY{p}{,}\PY{n}{np}\PY{o}{.}\PY{n}{zeros}\PY{p}{(}\PY{n}{x}\PY{o}{.}\PY{n}{shape}\PY{p}{[}\PY{l+m+mi}{1}\PY{p}{]}\PY{o}{+}\PY{l+m+mi}{1}\PY{p}{)}\PY{p}{,}\PY{l+m+mi}{1000}\PY{p}{,}\PY{l+m+mf}{0.75}\PY{p}{)}
\PY{n+nb}{print} \PY{p}{(}\PY{n}{ls\PYZus{}result}\PY{p}{[}\PY{l+m+mi}{1000}\PY{p}{]}\PY{p}{)}
\end{Verbatim}
\end{tcolorbox}

    \begin{Verbatim}[commandchars=\\\{\}]
[-0.32692597  0.22482821  0.1951127   0.23828183 -0.00534315  0.09740933
  0.0705666  -0.00171805  0.13957216  0.11758014  0.3303888 ]
    \end{Verbatim}

    \begin{tcolorbox}[breakable, size=fbox, boxrule=1pt, pad at break*=1mm,colback=cellbackground, colframe=cellborder]
\prompt{In}{incolor}{14}{\boxspacing}
\begin{Verbatim}[commandchars=\\\{\}]
\PY{n}{x\PYZus{}ray}\PY{o}{=}\PY{n+nb}{range}\PY{p}{(}\PY{l+m+mi}{0}\PY{p}{,}\PY{l+m+mi}{1001}\PY{p}{)}
\PY{n}{plt}\PY{o}{.}\PY{n}{plot}\PY{p}{(}\PY{n}{x\PYZus{}ray}\PY{p}{,}\PY{n}{ls\PYZus{}result}\PY{p}{)}
\PY{n}{plt}\PY{o}{.}\PY{n}{show}\PY{p}{(}\PY{p}{)}
\PY{c+c1}{\PYZsh{} 迭代过程中beta值的变化,横轴为迭代次数,纵轴为beta[i]的值}
\end{Verbatim}
\end{tcolorbox}

    \begin{center}
    \adjustimage{max size={0.9\linewidth}{0.9\paperheight}}{output_22_0.png}
    \end{center}
    { \hspace*{\fill} \\}
    
    \begin{tcolorbox}[breakable, size=fbox, boxrule=1pt, pad at break*=1mm,colback=cellbackground, colframe=cellborder]
\prompt{In}{incolor}{15}{\boxspacing}
\begin{Verbatim}[commandchars=\\\{\}]
\PY{n}{x\PYZus{}ray}\PY{o}{=}\PY{n+nb}{range}\PY{p}{(}\PY{l+m+mi}{0}\PY{p}{,}\PY{l+m+mi}{1000}\PY{p}{)}
\PY{n}{plt}\PY{o}{.}\PY{n}{plot}\PY{p}{(}\PY{n}{x\PYZus{}ray}\PY{p}{,}\PY{n}{ls\PYZus{}loss}\PY{p}{)}
\PY{n}{plt}\PY{o}{.}\PY{n}{show}\PY{p}{(}\PY{p}{)}
\PY{c+c1}{\PYZsh{} 迭代过程中损失函数值的变化,横轴为迭代次数,纵轴为损失函数值}
\end{Verbatim}
\end{tcolorbox}

    \begin{center}
    \adjustimage{max size={0.9\linewidth}{0.9\paperheight}}{output_23_0.png}
    \end{center}
    { \hspace*{\fill} \\}
    
    \subsection{牛顿法}\label{ux725bux987fux6cd5}

梯度下降法存在迭代次数过多,运算时间过长的问题。由于优化目标是一个convex
problem。我们可以使用牛顿法求解。首先我们计算损失函数的二阶导数。

\[
\begin{aligned}
\nabla^{2} L(\beta) &= \sum_{i=1}^{m} \frac{e^{X_i^T \beta}}{\left(1+e^{X_i^T \beta}\right)^2} X_i^T X_i \\
&=  \sum_{i=1}^{m} \frac{1}{2+e^{X_i^T \beta}  +e^{- X_i^T \beta} } X_i^T X_i
\end{aligned}
\]

    \begin{tcolorbox}[breakable, size=fbox, boxrule=1pt, pad at break*=1mm,colback=cellbackground, colframe=cellborder]
\prompt{In}{incolor}{16}{\boxspacing}
\begin{Verbatim}[commandchars=\\\{\}]
\PY{c+c1}{\PYZsh{} 损失函数二阶导数}
\PY{k}{def} \PY{n+nf}{logit\PYZus{}der2}\PY{p}{(}\PY{n}{x}\PY{p}{,}\PY{n}{y}\PY{p}{,}\PY{n}{b}\PY{p}{)}\PY{p}{:}
    \PY{n}{dim} \PY{o}{=} \PY{n}{x}\PY{o}{.}\PY{n}{shape}\PY{p}{[}\PY{l+m+mi}{0}\PY{p}{]}
    \PY{k}{return} \PY{p}{(}\PY{n+nb}{sum}\PY{p}{(}\PY{p}{[}  \PY{n}{np}\PY{o}{.}\PY{n}{matmul}\PY{p}{(}\PY{n}{x}\PY{p}{[}\PY{n}{i}\PY{p}{]}\PY{o}{.}\PY{n}{reshape}\PY{p}{(}\PY{n}{x}\PY{p}{[}\PY{n}{i}\PY{p}{]}\PY{o}{.}\PY{n}{shape}\PY{p}{[}\PY{l+m+mi}{0}\PY{p}{]}\PY{p}{,}\PY{l+m+mi}{1}\PY{p}{)}\PY{p}{,}\PY{n}{x}\PY{p}{[}\PY{n}{i}\PY{p}{]}\PY{o}{.}\PY{n}{reshape}\PY{p}{(}\PY{l+m+mi}{1}\PY{p}{,}\PY{n}{x}\PY{p}{[}\PY{n}{i}\PY{p}{]}\PY{o}{.}\PY{n}{shape}\PY{p}{[}\PY{l+m+mi}{0}\PY{p}{]}\PY{p}{)}\PY{p}{)}
                  \PY{o}{/}\PY{p}{(}\PY{n}{np}\PY{o}{.}\PY{n}{exp}\PY{p}{(}\PY{o}{\PYZhy{}} \PY{n}{np}\PY{o}{.}\PY{n}{vdot}\PY{p}{(}\PY{n}{x}\PY{p}{[}\PY{n}{i}\PY{p}{]}\PY{p}{,}\PY{n}{b}\PY{p}{)}\PY{p}{)} \PY{o}{+} \PY{n}{np}\PY{o}{.}\PY{n}{exp}\PY{p}{(}\PY{n}{np}\PY{o}{.}\PY{n}{vdot}\PY{p}{(}\PY{n}{x}\PY{p}{[}\PY{n}{i}\PY{p}{]}\PY{p}{,}\PY{n}{b}\PY{p}{)}\PY{p}{)} \PY{o}{+} \PY{l+m+mi}{2}\PY{p}{)} \PY{k}{for} \PY{n}{i} \PY{o+ow}{in} \PY{n+nb}{range}\PY{p}{(}\PY{n}{dim}\PY{p}{)}\PY{p}{]}\PY{p}{)}\PY{o}{/}\PY{n}{dim}\PY{p}{)}

\PY{k}{def} \PY{n+nf}{logit\PYZus{}newton}\PY{p}{(}\PY{n}{x}\PY{p}{,}\PY{n}{y}\PY{p}{,}\PY{n}{init}\PY{p}{,}\PY{n}{epoch}\PY{p}{)}\PY{p}{:}
    \PY{n}{b} \PY{o}{=} \PY{n}{np}\PY{o}{.}\PY{n}{ones}\PY{p}{(}\PY{n}{x}\PY{o}{.}\PY{n}{shape}\PY{p}{[}\PY{l+m+mi}{0}\PY{p}{]}\PY{p}{)}
    \PY{n}{x1} \PY{o}{=} \PY{n}{np}\PY{o}{.}\PY{n}{c\PYZus{}}\PY{p}{[}\PY{n}{b}\PY{p}{,}\PY{n}{x}\PY{p}{]}              \PY{c+c1}{\PYZsh{} 为自变量加上系数列}
    \PY{n}{beta} \PY{o}{=} \PY{n}{np}\PY{o}{.}\PY{n}{concatenate}\PY{p}{(}\PY{p}{(}\PY{n}{init}\PY{o}{.}\PY{n}{reshape}\PY{p}{(}\PY{l+m+mi}{1}\PY{p}{,}\PY{n}{init}\PY{o}{.}\PY{n}{shape}\PY{p}{[}\PY{l+m+mi}{0}\PY{p}{]}\PY{p}{)}\PY{p}{,}\PY{n}{np}\PY{o}{.}\PY{n}{zeros}\PY{p}{(}\PY{p}{(}\PY{n}{epoch}\PY{p}{,}\PY{n}{x1}\PY{o}{.}\PY{n}{shape}\PY{p}{[}\PY{l+m+mi}{1}\PY{p}{]}\PY{p}{)}\PY{p}{)}\PY{p}{)}\PY{p}{)}
    \PY{n}{loss} \PY{o}{=} \PY{n}{np}\PY{o}{.}\PY{n}{zeros}\PY{p}{(}\PY{n}{epoch}\PY{o}{+}\PY{l+m+mi}{1}\PY{p}{)}
    \PY{n}{loss}\PY{p}{[}\PY{l+m+mi}{0}\PY{p}{]} \PY{o}{=} \PY{n}{logit\PYZus{}loss}\PY{p}{(}\PY{n}{x1}\PY{p}{,}\PY{n}{y}\PY{p}{,}\PY{n}{init}\PY{p}{)}
    \PY{k}{for} \PY{n+nb}{iter} \PY{o+ow}{in} \PY{n+nb}{range}\PY{p}{(}\PY{n}{epoch}\PY{p}{)}\PY{p}{:}
        \PY{n}{second\PYZus{}der} \PY{o}{=}\PY{n}{logit\PYZus{}der2}\PY{p}{(}\PY{n}{x1}\PY{p}{,}\PY{n}{y}\PY{p}{,}\PY{n}{beta}\PY{p}{[}\PY{n+nb}{iter}\PY{p}{]}\PY{p}{)}
        \PY{n}{der\PYZus{}inv} \PY{o}{=} \PY{n}{np}\PY{o}{.}\PY{n}{linalg}\PY{o}{.}\PY{n}{inv}\PY{p}{(}\PY{n}{np}\PY{o}{.}\PY{n}{array}\PY{p}{(}\PY{n}{second\PYZus{}der}\PY{p}{,}\PY{n}{dtype}\PY{o}{=}\PY{l+s+s1}{\PYZsq{}}\PY{l+s+s1}{float}\PY{l+s+s1}{\PYZsq{}}\PY{p}{)}\PY{p}{)}
        \PY{n}{beta}\PY{p}{[}\PY{n+nb}{iter}\PY{o}{+}\PY{l+m+mi}{1}\PY{p}{]} \PY{o}{=} \PY{n}{beta}\PY{p}{[}\PY{n+nb}{iter}\PY{p}{]} \PY{o}{\PYZhy{}} \PY{n}{np}\PY{o}{.}\PY{n}{dot}\PY{p}{(}\PY{n}{der\PYZus{}inv}\PY{p}{,}\PY{n}{logit\PYZus{}der1}\PY{p}{(}\PY{n}{x1}\PY{p}{,}\PY{n}{y}\PY{p}{,}\PY{n}{beta}\PY{p}{[}\PY{n+nb}{iter}\PY{p}{]}\PY{p}{)}\PY{p}{)}
    \PY{k}{return} \PY{n}{beta}\PY{p}{,}\PY{n}{loss}
\end{Verbatim}
\end{tcolorbox}

    \begin{tcolorbox}[breakable, size=fbox, boxrule=1pt, pad at break*=1mm,colback=cellbackground, colframe=cellborder]
\prompt{In}{incolor}{ }{\boxspacing}
\begin{Verbatim}[commandchars=\\\{\}]
\PY{n}{对全部数据进行牛顿法逻辑回归计算}\PY{p}{,} \PY{n}{得到结果如下}\PY{p}{,} \PY{n}{发现收敛速度很快}\PY{err}{。}
\end{Verbatim}
\end{tcolorbox}

    \begin{tcolorbox}[breakable, size=fbox, boxrule=1pt, pad at break*=1mm,colback=cellbackground, colframe=cellborder]
\prompt{In}{incolor}{17}{\boxspacing}
\begin{Verbatim}[commandchars=\\\{\}]
\PY{n}{x1}\PY{p}{,}\PY{n}{y} \PY{o}{=} \PY{n}{read\PYZus{}sig}\PY{p}{(}\PY{n}{data}\PY{p}{)}
\PY{n}{x} \PY{o}{=} \PY{n}{normalize}\PY{p}{(}\PY{n}{x1}\PY{p}{)}
\PY{n}{newton\PYZus{}result}\PY{p}{,} \PY{n}{newton\PYZus{}loss} \PY{o}{=} \PY{n}{logit\PYZus{}newton}\PY{p}{(}\PY{n}{x}\PY{p}{,}\PY{n}{y}\PY{p}{,}\PY{n}{np}\PY{o}{.}\PY{n}{array}\PY{p}{(}\PY{p}{[}\PY{l+m+mi}{0}\PY{p}{,}\PY{l+m+mi}{0}\PY{p}{,}\PY{l+m+mi}{0}\PY{p}{,}\PY{l+m+mi}{0}\PY{p}{,}\PY{l+m+mi}{0}\PY{p}{,}\PY{l+m+mi}{0}\PY{p}{,}\PY{l+m+mi}{0}\PY{p}{,}\PY{l+m+mi}{0}\PY{p}{,}\PY{l+m+mi}{0}\PY{p}{,}\PY{l+m+mi}{0}\PY{p}{,}\PY{l+m+mi}{0}\PY{p}{]}\PY{p}{)}\PY{p}{,}\PY{l+m+mi}{40}\PY{p}{)}
\PY{n+nb}{print} \PY{p}{(}\PY{n}{newton\PYZus{}result}\PY{p}{[}\PY{l+m+mi}{10}\PY{p}{]}\PY{p}{)}
\end{Verbatim}
\end{tcolorbox}

    \begin{Verbatim}[commandchars=\\\{\}]
[  0.38755791 -34.6008646   76.36750292   7.56697541  -2.10969107
  31.01540005 -11.97725587   4.36794273  12.81851241   9.56045522
  15.77466697]
    \end{Verbatim}

    \begin{tcolorbox}[breakable, size=fbox, boxrule=1pt, pad at break*=1mm,colback=cellbackground, colframe=cellborder]
\prompt{In}{incolor}{18}{\boxspacing}
\begin{Verbatim}[commandchars=\\\{\}]
\PY{n}{x\PYZus{}ray}\PY{o}{=}\PY{n+nb}{range}\PY{p}{(}\PY{l+m+mi}{0}\PY{p}{,}\PY{l+m+mi}{41}\PY{p}{)}
\PY{n}{plt}\PY{o}{.}\PY{n}{plot}\PY{p}{(}\PY{n}{x\PYZus{}ray}\PY{p}{,}\PY{n}{newton\PYZus{}result}\PY{p}{)}
\PY{n}{plt}\PY{o}{.}\PY{n}{show}\PY{p}{(}\PY{p}{)}
\end{Verbatim}
\end{tcolorbox}

    \begin{center}
    \adjustimage{max size={0.9\linewidth}{0.9\paperheight}}{output_28_0.png}
    \end{center}
    { \hspace*{\fill} \\}
    
    \section{模型测试}\label{ux6a21ux578bux6d4bux8bd5}

我们使用留出法对模型进行简单测试,我们调用sklearn的方法将数据集随机分为75\%的训练集和25\%的测试集,调用牛顿法逻辑回归模型,训练结果如下所示。

    \begin{tcolorbox}[breakable, size=fbox, boxrule=1pt, pad at break*=1mm,colback=cellbackground, colframe=cellborder]
\prompt{In}{incolor}{48}{\boxspacing}
\begin{Verbatim}[commandchars=\\\{\}]
\PY{k+kn}{from} \PY{n+nn}{sklearn}\PY{n+nn}{.}\PY{n+nn}{model\PYZus{}selection} \PY{k+kn}{import} \PY{n}{train\PYZus{}test\PYZus{}split}
\PY{n}{X\PYZus{}train}\PY{p}{,} \PY{n}{X\PYZus{}test}\PY{p}{,} \PY{n}{y\PYZus{}train}\PY{p}{,} \PY{n}{y\PYZus{}test} \PY{o}{=} \PY{n}{train\PYZus{}test\PYZus{}split}\PY{p}{(}\PY{n}{x}\PY{p}{,} \PY{n}{y}\PY{p}{,} \PY{n}{test\PYZus{}size}\PY{o}{=}\PY{l+m+mf}{0.25}\PY{p}{,} \PY{n}{random\PYZus{}state}\PY{o}{=}\PY{l+m+mi}{42}\PY{p}{)}
\PY{n}{train\PYZus{}result}\PY{p}{,}\PY{n}{train\PYZus{}loss} \PY{o}{=} \PY{n}{logit\PYZus{}newton}\PY{p}{(}\PY{n}{X\PYZus{}train}\PY{p}{,}\PY{n}{y\PYZus{}train}\PY{p}{,}\PY{n}{np}\PY{o}{.}\PY{n}{array}\PY{p}{(}\PY{p}{[}\PY{l+m+mi}{0}\PY{p}{,}\PY{l+m+mi}{0}\PY{p}{,}\PY{l+m+mi}{0}\PY{p}{,}\PY{l+m+mi}{0}\PY{p}{,}\PY{l+m+mi}{0}\PY{p}{,}\PY{l+m+mi}{0}\PY{p}{,}\PY{l+m+mi}{0}\PY{p}{,}\PY{l+m+mi}{0}\PY{p}{,}\PY{l+m+mi}{0}\PY{p}{,}\PY{l+m+mi}{0}\PY{p}{,}\PY{l+m+mi}{0}\PY{p}{]}\PY{p}{)}\PY{p}{,}\PY{l+m+mi}{40}\PY{p}{)}
\PY{n+nb}{print}\PY{p}{(}\PY{n}{train\PYZus{}result}\PY{p}{[}\PY{o}{\PYZhy{}}\PY{l+m+mi}{1}\PY{p}{:}\PY{p}{]}\PY{p}{)}
\end{Verbatim}
\end{tcolorbox}

    \begin{Verbatim}[commandchars=\\\{\}]
[[  0.28598197 -31.55924357  70.88713246   8.54281063  -2.25819591
   29.13817011 -11.75582428   3.75706483  13.57355246   8.8459486
   14.78123584]]
    \end{Verbatim}

    \begin{tcolorbox}[breakable, size=fbox, boxrule=1pt, pad at break*=1mm,colback=cellbackground, colframe=cellborder]
\prompt{In}{incolor}{63}{\boxspacing}
\begin{Verbatim}[commandchars=\\\{\}]
\PY{k}{def} \PY{n+nf}{test\PYZus{}model}\PY{p}{(}\PY{n}{x1}\PY{p}{,}\PY{n}{b}\PY{p}{)}\PY{p}{:}
    \PY{n}{one} \PY{o}{=} \PY{n}{np}\PY{o}{.}\PY{n}{ones}\PY{p}{(}\PY{n}{x1}\PY{o}{.}\PY{n}{shape}\PY{p}{[}\PY{l+m+mi}{0}\PY{p}{]}\PY{p}{)}
    \PY{n}{x} \PY{o}{=} \PY{n}{np}\PY{o}{.}\PY{n}{c\PYZus{}}\PY{p}{[}\PY{n}{one}\PY{p}{,}\PY{n}{x1}\PY{p}{]}
    \PY{n}{x}\PY{o}{=}\PY{n}{np}\PY{o}{.}\PY{n}{array}\PY{p}{(}\PY{n}{x}\PY{p}{,}\PY{n}{dtype}\PY{o}{=}\PY{n}{np}\PY{o}{.}\PY{n}{float64}\PY{p}{)}
    \PY{n}{b}\PY{o}{=}\PY{n}{np}\PY{o}{.}\PY{n}{array}\PY{p}{(}\PY{n}{b}\PY{p}{,}\PY{n}{dtype}\PY{o}{=}\PY{n}{np}\PY{o}{.}\PY{n}{float64}\PY{p}{)}
    \PY{n}{sigmoid} \PY{o}{=} \PY{l+m+mi}{1}\PY{o}{/}\PY{p}{(}\PY{l+m+mi}{1}\PY{o}{+}\PY{n}{np}\PY{o}{.}\PY{n}{exp}\PY{p}{(}\PY{o}{\PYZhy{}} \PY{n}{np}\PY{o}{.}\PY{n}{dot}\PY{p}{(}\PY{n}{x}\PY{p}{,}\PY{n}{b}\PY{o}{.}\PY{n}{T}\PY{p}{)}\PY{p}{)}\PY{p}{)}
    \PY{k}{return} \PY{p}{(}\PY{n}{np}\PY{o}{.}\PY{n}{sign}\PY{p}{(}\PY{n}{sigmoid}\PY{o}{\PYZhy{}}\PY{l+m+mf}{0.5}\PY{p}{)}\PY{o}{+}\PY{l+m+mi}{1}\PY{p}{)}\PY{o}{/}\PY{l+m+mi}{2}

\PY{n}{np}\PY{o}{.}\PY{n}{set\PYZus{}printoptions}\PY{p}{(}\PY{n}{threshold}\PY{o}{=}\PY{l+m+mi}{1}\PY{p}{)}
\PY{n}{test\PYZus{}results} \PY{o}{=} \PY{n}{np}\PY{o}{.}\PY{n}{c\PYZus{}}\PY{p}{[}\PY{n}{test\PYZus{}model}\PY{p}{(}\PY{n}{X\PYZus{}test}\PY{p}{,}\PY{n}{train\PYZus{}result}\PY{p}{[}\PY{o}{\PYZhy{}}\PY{l+m+mi}{1}\PY{p}{:}\PY{p}{]}\PY{p}{)}\PY{p}{,}\PY{n}{y\PYZus{}test}\PY{p}{]}
\PY{n+nb}{print} \PY{p}{(}\PY{n}{test\PYZus{}results}\PY{p}{)}\PY{c+c1}{\PYZsh{} 测试集结果与真实值比较}
\end{Verbatim}
\end{tcolorbox}

    \begin{Verbatim}[commandchars=\\\{\}]
[[0. 0.]
 [1. 1.]
 [1. 1.]
 {\ldots}
 [0. 0.]
 [1. 1.]
 [0. 0.]]
    \end{Verbatim}

    我们对测试集和真实结果进行比较, 发现逻辑回归对该数据集预测效果较好。

    \begin{tcolorbox}[breakable, size=fbox, boxrule=1pt, pad at break*=1mm,colback=cellbackground, colframe=cellborder]
\prompt{In}{incolor}{70}{\boxspacing}
\begin{Verbatim}[commandchars=\\\{\}]
\PY{k}{def} \PY{n+nf}{test\PYZus{}analysis}\PY{p}{(}\PY{n}{result}\PY{p}{)}\PY{p}{:}
    \PY{n}{TP}\PY{p}{,} \PY{n}{FP}\PY{p}{,} \PY{n}{TN}\PY{p}{,} \PY{n}{FN} \PY{o}{=} \PY{l+m+mi}{0}\PY{p}{,} \PY{l+m+mi}{0}\PY{p}{,} \PY{l+m+mi}{0}\PY{p}{,} \PY{l+m+mi}{0}
    \PY{k}{for} \PY{n}{i} \PY{o+ow}{in} \PY{n+nb}{range}\PY{p}{(}\PY{n}{result}\PY{o}{.}\PY{n}{shape}\PY{p}{[}\PY{l+m+mi}{0}\PY{p}{]}\PY{p}{)}\PY{p}{:}
        \PY{k}{if} \PY{p}{(}\PY{n}{result}\PY{p}{[}\PY{n}{i}\PY{p}{]} \PY{o}{==} \PY{p}{[}\PY{l+m+mi}{0}\PY{p}{,}\PY{l+m+mi}{0}\PY{p}{]}\PY{p}{)}\PY{o}{.}\PY{n}{all}\PY{p}{(}\PY{p}{)}\PY{p}{:}
            \PY{n}{TN} \PY{o}{+}\PY{o}{=} \PY{l+m+mi}{1}
        \PY{k}{if} \PY{p}{(}\PY{n}{result}\PY{p}{[}\PY{n}{i}\PY{p}{]} \PY{o}{==} \PY{p}{[}\PY{l+m+mi}{0}\PY{p}{,}\PY{l+m+mi}{1}\PY{p}{]}\PY{p}{)}\PY{o}{.}\PY{n}{all}\PY{p}{(}\PY{p}{)}\PY{p}{:}
            \PY{n}{FN} \PY{o}{+}\PY{o}{=} \PY{l+m+mi}{1}
        \PY{k}{if} \PY{p}{(}\PY{n}{result}\PY{p}{[}\PY{n}{i}\PY{p}{]} \PY{o}{==} \PY{p}{[}\PY{l+m+mi}{1}\PY{p}{,}\PY{l+m+mi}{1}\PY{p}{]}\PY{p}{)}\PY{o}{.}\PY{n}{all}\PY{p}{(}\PY{p}{)}\PY{p}{:}
            \PY{n}{TP} \PY{o}{+}\PY{o}{=} \PY{l+m+mi}{1}
        \PY{k}{if} \PY{p}{(}\PY{n}{result}\PY{p}{[}\PY{n}{i}\PY{p}{]} \PY{o}{==} \PY{p}{[}\PY{l+m+mi}{1}\PY{p}{,}\PY{l+m+mi}{0}\PY{p}{]}\PY{p}{)}\PY{o}{.}\PY{n}{all}\PY{p}{(}\PY{p}{)}\PY{p}{:}
            \PY{n}{FP} \PY{o}{+}\PY{o}{=} \PY{l+m+mi}{1}
    \PY{n+nb}{print} \PY{p}{(}\PY{n}{TN}\PY{p}{,}\PY{n}{FN}\PY{p}{,}\PY{n}{TP}\PY{p}{,}\PY{n}{FP}\PY{p}{)}
    \PY{n+nb}{print} \PY{p}{(}\PY{l+s+s2}{\PYZdq{}}\PY{l+s+s2}{查准率P: }\PY{l+s+s2}{\PYZdq{}}\PY{p}{,} \PY{n}{TP}\PY{o}{/}\PY{p}{(}\PY{n}{TP}\PY{o}{+}\PY{n}{FP}\PY{p}{)}\PY{p}{)}
    \PY{n+nb}{print} \PY{p}{(}\PY{l+s+s2}{\PYZdq{}}\PY{l+s+s2}{查全率R: }\PY{l+s+s2}{\PYZdq{}}\PY{p}{,} \PY{n}{TP}\PY{o}{/}\PY{p}{(}\PY{n}{TP}\PY{o}{+}\PY{n}{FN}\PY{p}{)}\PY{p}{)}
    \PY{n+nb}{print} \PY{p}{(}\PY{l+s+s2}{\PYZdq{}}\PY{l+s+s2}{F1: }\PY{l+s+s2}{\PYZdq{}}\PY{p}{,} \PY{l+m+mi}{2}\PY{o}{*}\PY{n}{TP}\PY{o}{/}\PY{p}{(}\PY{l+m+mi}{2}\PY{o}{*}\PY{n}{TP}\PY{o}{+}\PY{n}{FN}\PY{o}{+}\PY{n}{FP}\PY{p}{)}\PY{p}{)}
    \PY{k}{return}

\PY{n}{test\PYZus{}analysis}\PY{p}{(}\PY{n}{test\PYZus{}results}\PY{p}{)}
\end{Verbatim}
\end{tcolorbox}

    \begin{Verbatim}[commandchars=\\\{\}]
88 2 52 1
查准率P:  0.9811320754716981
查全率R:  0.9629629629629629
F1:  0.9719626168224299
    \end{Verbatim}

    \section{问题和讨论}\label{ux95eeux9898ux548cux8ba8ux8bba}

由于时间精力和计算资源有限,本次实验中遇到了以下问题有待解决。 

1.
如何看待sklearn标准库中的逻辑回归模型、梯度下降模型、牛顿法模型得到的参数值有较大出入?

2.
实验中遇到了大量浮点数运算溢出的问题,经过对原数据不同程度normalize的尝试才得以避免,这样的操作合理吗?

3.
在梯度下降的模型中,如何选择合适的步长?经过尝试,发现使用固定步长存在不收敛的可能性,而使用backtracking
line
search等方法虽然理论上有可以收敛,但可能导致步长锐减太快达到浮点数运算的极限,是否存在一些调整参数、选择方法上的原则?


    % Add a bibliography block to the postdoc
    
    
    
\end{document}
