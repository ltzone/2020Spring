%%%%%%%%%%%%%%%%%%%%%%%%%%%%%%%%%%%%%%%%%%%%%%%%%%%%
%%%%%%%%%%%%%%%%%%%%%%%%%%%%EXERCISE 1%%%%%%%%%%%%%%
%%%%%%%%%%%%%%%%%%%%%%%%%%%%%%%%%%%%%%%%%%%%%%%%%%%%
\begin{exercise} [Coin flips] { A fair coin is flipped until the first head occurs. Let $X$ denote the number of flips required. 
\begin{enumerate}
\item Find the entropy $H(X)$ in bits. The following expressions may be useful:
\begin{equation}
\sum_{n=0}^{\infty} r^{n}=\frac{1}{1-r}, \quad \sum_{n=0}^{\infty} n r^{n}=\frac{r}{(1-r)^{2}}
\end{equation}

\item A random variable $X$ is drawn according to this distribution. Find an “efficient” sequence of yes–no questions of the form, “Is $X$ contained in the set $S$?” Compare $H(X)$ to the expected number of questions required to determine $X$.
\end{enumerate}
}


\begin{solution}
\par{~}
\end{solution}
\end{exercise}

%%%%%%%%%%%%%%%%%%%%%%%%%%%%%%%%%%%%%%%%%%%%%%%%%%%%
%%%%%%%%%%%%%%%%%%%%%%%%%%%%EXERCISE 2%%%%%%%%%%%%%%
%%%%%%%%%%%%%%%%%%%%%%%%%%%%%%%%%%%%%%%%%%%%%%%%%%%%
\begin{exercise}[Zero conditional entropy] {Show that if $H(Y|X) = 0$, then $Y$ is a function of $X$ [i.e., for all $x$ with $p(x) > 0$, there is only one possible value of $y$ with $p(x, y) > 0$].}

\begin{proof}
\par{~}
\end{proof}
\end{exercise}


%%%%%%%%%%%%%%%%%%%%%%%%%%%%%%%%%%%%%%%%%%%%%%%%%%%%
%%%%%%%%%%%%%%%%%%%%%%%%%%%%EXERCISE 3%%%%%%%%%%%%%%
%%%%%%%%%%%%%%%%%%%%%%%%%%%%%%%%%%%%%%%%%%%%%%%%%%%%
\begin{exercise}[Coin weighing] {Suppose that one has $n$ coins, among which there may or may not be one counterfeit coin. If there is a counterfeit coin, it may be either heavier or lighter than the other coins. The coins are to be weighed by a balance.
\begin{enumerate}
\item Find an upper bound on the number of coins $n$ so that $k$ weighings will find the counterfeit coin (if any) and correctly declare it to be heavier or lighter.
\item (Difficult) What is the coin-weighing strategy for $k = 3$ weighings and 12 coins?
\end{enumerate}
}
\begin{solution}
\par{~}
\end{solution}
\end{exercise}