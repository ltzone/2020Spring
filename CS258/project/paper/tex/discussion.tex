\section{Discussion}

In this project, we first revisited I-Measure, a correspondence between Shannon's information and set theory. Then we introduced redundancy measure, together with its decomposed form, partial information, a non-negative measure in describing the redundancy a multivariate collection of sources provide for a particular random variable.

Inspired by I-Measure, we formally defined ``PI-Measure'' in the scope of set theory to formally describe how partial information diagrams are formed. We formulate the correspondence rules between PI-Measure and gave out its correspondence with partial information and redundancy measure. Then we made some simplifications to eliminate redundant fields. We've also found a strategy to find alternative expressions on partial information from a pure set-theoretic perspective. Finally, we calculated the partial information for some simple examples to compare our proposed strategy with the existing one.

We hope that our formulation can help clarify and substantiate how partial information diagrams are generated and can be used.
