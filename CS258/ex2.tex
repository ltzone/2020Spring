\begin{exercise} {Show that $D(p \| q)=0$ if and only if $p(x)=q(x)$.}
\begin{proof}
\begin{equation}
  \begin{aligned}
    - D(p(x)\| q(x)) &= \sum_{x\in \mathcal{X}} p(x) \log \frac{q(x)}{p(x)} & \\
    &\le \log(\sum p(x) \frac{q(x)}{p(x)}) & \text{(By Concavity of log(x))} \\
    &= \log(\sum q(x)) \le \log 1 = 0  &
  \end{aligned}
\end{equation}

The first equality holds if and only if $$\frac{q(x)}{p(x)}=k, \text{ for every } x \in \mathcal{X} \text{ such that } p(x)>0 $$

The second equality holds if and only if there exists no $x$ such that $p(x)=0$ while $q(x)>0$. 
Since we have $\sum p(x) = 1$, we know that $$\frac{q(x_1)}{p(x_1)} = \frac{q(x_2)}{p(x_2)} = \cdots = \frac{\sum{q(x)}}{\sum p(x)}. $$

By the second condition we know that $\sum q(x) = \sum p(x) = 1$. Hence $p(x) = q(x)$ for all $x\in \mathcal{X}$.

\end{proof}
\label{ex2-1}
\end{exercise}


\begin{exercise} {Show that $I(X;Y) \ge 0$, with equality if and only if $X$ and $Y$ are independent.}
\begin{proof}
\begin{equation}
  \begin{aligned}
    -I(X;Y) &= \sum_{(x,y)\in \mathcal{X}\star \mathcal{Y}} p(x,y) \log {\frac{p(x)p(y)}{p(x,y)}} \\
    &\le \log \left( \sum_{(x,y)\in \mathcal{X} \star \mathcal{Y}} p(x,y) \frac{p(x)p(y)}{p(x,y)}\right) \\
    &=  \log{ \sum_{(x,y)\in \mathcal{X} \star \mathcal{Y} } {p(x)p(y)}} = \log \left(\sum_{x\in \mathcal{X}}p(x) \sum_{y \in \mathcal{Y}} p(y) \right) = \log 1 = 0
  \end{aligned}
\end{equation}

The equality holds if and only if $$\frac{p(x,y)}{p(x)p(y)}=k, \text{ for every } (x,y) \in \mathcal{X} \star \mathcal{Y} \text{ such that } p(x,y)>0$$

Since $\sum p(x,y) = 1$, we know that $k=1$. That is to say, $p(x,y) = p(x)p(y)$ for every possible $x,y$. $X$ and $Y$ are independent.

\end{proof}
\label{ex2-2}
\end{exercise}


\begin{exercise} {Show that $D(p(y|x)\| q(y|x))\ge 0$ with equality if and only if $p(y|x)=q(y|x)$ for all $x$ and $y$ such that $p(x)>0$.}
\begin{proof}
  \begin{equation}
    \begin{aligned}
      -D(p(y|x)\| q(y|x)) &= \sum_{x\in \mathcal{X}} p(x) \sum_{y\in \mathcal{Y}}p(y|x)\log \frac{q(y|x)}{p(y|x)} \\
      &\le \sum_{x\in \mathcal{X}} p(x) \log \sum_{y\in \mathcal{Y}} q(y|x) \\
      &\le \sum_{x\in \mathcal{X}} p(x) \log 1 = 0
    \end{aligned}
  \end{equation}

  The first equality holds if and only if $$\frac{q(y|x)}{p(y|x)}=k, \text{ for every } y \in \mathcal{Y} \text{ such that } p(y|x)>0 \text{ given } p(x)>0 \text{ with } x \in \mathcal{X} $$

  The second equality holds if and only if given $p(x)>0$, there exists no $y$ such that $p(y|x)=0$ while $q(y|x)>0$.

  Since we have $\sum p(y|x) = 1$, we know that $$\frac{q(y_1|x)}{p(y_1|x)} = \frac{q(y_2|x)}{p(y_2|x)} = \cdots = \frac{\sum{q(y|x)}}{\sum p(y|x)}. $$

  It follows from the second condition that $\sum q(y|x) = \sum p(y|x) = 1$. Hence the equality holds if and only if $p(y|x) = q(y|x)$ for all $p(x)>0$.

\end{proof}
\end{exercise}


\begin{exercise} {Show that $I(X;Y|Z)\ge 0$ with equality if and only if $X$ and $Y$ are conditionally independent given $Z$.}
\begin{proof}
\begin{equation}
  \begin{aligned}
    -I(X;Y|Z) &= \sum_{(x,y,z)\in \mathcal{X}\star \mathcal{Y} \star \mathcal{Z}} p(x,y,z) \log {\frac{p(x|z)p(y|z)}{p(x,y|z)}} \\
    &\le \sum_{z\in \mathcal{Z}} p(z) \log \sum_{(x,y)\in \mathcal{X} \star \mathcal{Y} } p(x,y|z) \frac{p(x|z)p(y|z)}{p(x,y|z)} \\
    &= \sum_{z\in \mathcal{Z}} p(z)  \log \left(\sum_{x\in \mathcal{X}}p(x|z) \sum_{y \in \mathcal{Y}} p(y|z) \right) \\
    &= \sum_{z\in \mathcal{Z}} p(z) \log 1 = 0
  \end{aligned}
\end{equation}

The equality holds if and only if $$\frac{p(x,y|z)}{p(x|z)p(y|z)}=k, \text{ for every } (x,y) \in \mathcal{X} \star \mathcal{Y} \text{ such that } p(x,y)>0 \text{ given } p(z)>0 \text{ with } z\in \mathcal{Z}$$

Since $\sum p(x,y|z) = 1$, we know that $k=1$. That is to say, $p(x,y|z) = p(x|z)p(y|z)$ for every possible $x,y$ given $p(z)>0$. Therefore the equality holds if and only if $X$ and $Y$ are independent given $Z$.

\end{proof}
\end{exercise}


\begin{exercise} {Let $u(x)=\frac{1}{|\mathcal{X}|}$ be the uniform probability mass fuction over $X$, and let $p(x)$ be the probability mass function for $X$, Then
  $$0\le D(p\| u)= \log |\mathcal{X}|-H(X)$$}
\begin{proof}
  
  From Exercise \ref{ex2-1} we know $D(p\| u)\ge 0$. By definition of mutual entropy we have

  \begin{equation}
    \begin{aligned}
      D(p\| u) &= \sum_{x\in \mathcal{X}}p(x) \log{\frac{p(x)}{u(x)}} = \sum_{x\in \mathcal{X}}p(x) \log{|\mathcal{X}|p(x)} \\
      &= \log{|\mathcal{X}|}\sum_{x\in \mathcal{X}}p(x) - \sum_{x\in \mathcal{X}} p(x) \log p(x) =  \log |\mathcal{X}|-H(X)
    \end{aligned}
  \end{equation}
\end{proof}
\end{exercise}

\begin{exercise} [Conditioning reduces entropy] {Show that $$H(X|Y) \le H(X)$$
  with equality if and only if $X$ and $Y$ are independent. }
\begin{proof}
\par{~}
We know that $I(X;Y) = H(X) - H(X|Y)$. From Exercise \ref{ex2-2} we know that $I(X;Y) \ge 0$. It follows that $H(X|Y) \le H(X)$ with equality if and only if $X$ and $Y$ are independent.
\end{proof}
\end{exercise}
  